%========================================================
%Proceso
%========================================================

%========================================================
% Descripción general del proceso
%-----------------------------------------------
\begin{Proceso}{P4.2}{Dar de Baja Lector} {
  
  %-------------------------------------------
  %Resumen



  Proceso que realiza el \cdtRef{Actor:Bibliotecario}{Bibliotecario} con el fin de dar de baja a un \cdtRef{Actor:Lector}{Lector} ya sea alumno o docente del sistema siguiendo reglas de negocio que permiten que el usuario sea dado de baja cuando se encuentre libre de adeudos y multas.
  
Como primer paso se tendrá que validar al lector mediante su credencial de manera que se compruebe que este registrado en el sistema.
Si el lector al que se dará de baja tiene adeudos o multas pendientes, el sistema se encargará de notificar al bibliotecario posteriormente se mostraran los adeudos y multas que corresponden al lector.


  Para que proceda la baja, el \cdtRef{Actor:Lector}{Lector} tiene la responsabilidad tanto de pagar sus multas y devolver el material que tiene pendiente, de lo contrario la baja no procederá.
El sistema termina el proceso de dar de baja. 

  %-------------------------------------------
  %Diagrama del proceso

  \noindent La Figura \cdtRefImg{P4.2}{Dar de baja Lector} muestra las actividades que se realizan para llevar a cabo el proceso descrito anteriormente.

  \Pfig[0.95]{./procesos/C4/Images/GU4_2-DarDeBajaLector.png}{P4.2}{Dar de Baja Lector}

} {P4.2:Dar de Baja Lector}

  %-------------------------------------------
  %Elementos del proceso

  \UCitem{Actores} { %Actores
    \cdtRef{Actor:Lector}{Lector} y \cdtRef{Actor:Bibliotecario}{Bibliotecario}.
  }

  \UCitem{Objetivo} { %Objetivo
    Dar de baja a un lector verificando que no tenga adeudos o multas en su historial.
  }

  \UCitem{Insumos de entrada} { %Insumos de entrada
  	\begin{UClist}
  		\UCli Id del Lector
    \end {UClist}
  }
  
  \UCitem{Proveedores} { %Proveedores
    \cdtRef{Actor:Lector}{Lector}
  }

  \UCitem{Productos de salida} { %Productos de salida
    \begin{UClist}
      \UCli Notificación \cdtIdRef{MSJ4.5}{Operación Exitosa}.
      \UCli Notificación \cdtIdRef{MSJ4.1}{Lector No registrado}.
            \UCli Notificación \cdtIdRef{MSJ4.2}{Tienes Adeudos pendientes}.
            \UCli Tabla que contiene las multas pendientes del \cdtRef{Actor:Lector}{Lector}.
\UCli Notificación \cdtIdRef{MSJ4.4}{Tienes Multas pendientes}.            
\UCli Tabla que contiene los adeudos pendientes del \cdtRef{Actor:Lector}{Lector}.

    \end{UClist}
  }

  \UCitem{Cliente} { %Cliente
    \cdtRef{Actor:Lector}{Lector}
  }

  \UCitem{Mecanismo de medición} { %Mecanismo de medición
    \begin{UClist}
      \UCli Respuesta inmediata
    \end{UClist}
  }
  \UCitem{Interrelación con otros procesos} { %Interrelación con otros procesos
    \cdtIdRef{P4.4}{Consulta Historial de lector}
  }


\end{Proceso}

%========================================================
%Descripción de tareas
%-----------------------------------------------
\begin{PDescripcion}

  %Actor: Aspirante
  \Ppaso Lector

    \begin{enumerate}

      %Tarea a
      \Ppaso[\itarea] \cdtLabelTask{T1-P4.2:Lector}{Pagar Multas.}Notifica al lector que tiene multas pendientes y debe pagarlas de lo contrario no se puede dar de baja.
      %Tarea b
      \Ppaso[\itarea] \cdtLabelTask{T2-P4.2:Lector}{Devuelve el material.}Devuelve el material. Notifica al lector que aún tiene devoluciones pendientes y debe devolver el material, de lo contrario la baja no procederá.

    \end{enumerate}

  %Actor: SAEV2.0
  \Ppaso Bibliotecario

    \begin{enumerate}

      %Tarea a
      \Ppaso[\itarea] \cdtLabelTask{T1-P4.2:Bibliotecario}{Validar al Lector.} Para que el sistema pueda validar al lector debe verificar la Regla de negocio RN4.4 Identificación del lector. Si la regla de negocio se cumple pasa a la tarea \cdtRefTask{T3-P4.2:Bibliotecario}{Dar de baja Lector}, si no se cumple la regla de negocio el sistema pasa a la tarea \cdtRefTask{T2-P4.2:Bibliotecario}{Autenticar Lector}.

%referenciar tareas \cdtRefTask{T2-P0.1:SAEV2.0}{Notifica solicitud fuera de periodo de registro.}


      %Tarea b
      \Ppaso[\itarea] \cdtLabelTask{T2-P4.2:Bibliotecario}{Autenticar Lector.} El sistema autenticará al lector a partir del ID del mismo para comprobar si el lector está registrado de no estar registrado muestra la Notificación \cdtIdRef{MSJ4.1}{Lector No registrado}.
      %Tarea c
      \Ppaso[\itarea] \cdtLabelTask{T3-P4.2:Bibliotecario}{Dar de baja Lector.}Para que el sistema proceda con la baja del Lector deberá verificar las reglas de negocio RN4.5 El lector tiene multas y RN4.6 El lector tiene adeudos. Si las reglas de negocio se cumplen se pasa a la tarea \cdtRefTask{T6-P4.2:Bibliotecario}{Dar de baja Lector}, si no se cumple la RN4.5 se pasa a la tarea \cdtRefTask{T4-P4.2:Bibliotecario}{Generar reporte de multa}, si no se cumple la RN4.5 se pasa a la tarea \cdtRefTask{T5-P4.2:Bibliotecario}{Generar reporte de adeudo(s)}.

      %Tarea d
      \Ppaso[\itarea] \cdtLabelTask{T4-P4.2:Bibliotecario}{Generar reporte de multa.} El sistema genera la notificación NS3 al igual que la tabla indicando las multas que tiene pendientes a pagar para que sean mostrados al lector pasando a la tarea \cdtRefTask{T1-P4.2:Lector}{Pagar multa(s)}. 

      %Tarea e
      \Ppaso[\itarea] \cdtLabelTask{T5-P4.2:Bibliotecario}{Genera reporte de adeudo(s).} El sistema genera la notificación NS4 al igual que la tabla indicando los adeudos que tiene pendientes a devolver para que sean mostrados al lector pasando a la tarea \cdtRefTask{T2-P4.2:Lector}{Devueve el material}.

      %Tarea f
      \Ppaso[\itarea] \cdtLabelTask{T6-P4.2:Bibliotecario}{Dar de baja Lector.} El sistema dará de baja al lector y mostrará la notificación \cdtIdRef{MSJ4.5}{Operación Exitosa}.

    \end{enumerate}

\end{PDescripcion}
