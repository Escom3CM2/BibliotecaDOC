%-----------------------------------------------------------------------------------------------------------------------------------------------
%========================================================
%Proceso Actualizar datos de un Alumno
%========================================================

%========================================================
% Descripción general del proceso
%-----------------------------------------------
\begin{Proceso}{P2.1}{Adquisición de Libros} {
  
  %-------------------------------------------
  %Resumen

  Proceso que realiza el \cdtRef{Actor:Bibliotecario}{Bibliotecario} con la finalidad de obtener nuevos ejemplares de libros que se requieren de acuerdo a las asignaturas de cada academia y sujeto al presupuesto establecido por Recursos Materiales. 
  
  Cada jefe de academia es encargado de ques los profesores de realicen una propuesta de libros para evaluarlas de acuerdo a su nivel de importancia y hacer un selección que es enviada al bibliotecario para que éste último realice el pedido. Al efectuar el pago se espera la entrega, se verifica el pedido y se lleva a cabo el proceso de registro y preparación de los libros para su disponibilidad en biblioteca.
  
  %-------------------------------------------
  %Diagrama del proceso
  \newpage
  \noindent La Figura \cdtRefImg{P2.1}{Adquirir Libros} muestra las actividades que se realizan para llevar a cabo el proceso descrito anteriormente.

  \Pfig[0.85]{./procesos/C2/Images/adquirirLibros.png}{P2.1}{Adquirir Libros}

} {P2.1:Adquirir Libros}

  %-------------------------------------------
  %Elementos del proceso
  \UCitem{Actores} { %Actores
    \cdtRef{Actor:Bibliotecario}{Bibliotecario}
  }

  \UCitem{Objetivo} { %Objetivo
    Obtener nuevos ejemplares de libros para la biblioteca.
  }

  \UCitem{Insumos de entrada} { %Insumos de entrada
  	\begin{UClist}
  		\UCli Datos del Formulario \cdtIdRef{F2.1}{Petición de Libros}. 
     	
    \end {UClist}
  }
  
  \UCitem{Proveedores} { %Proveedores
    \cdtRef{Actor:Bibliotecario}{Bibliotecario}
  }

  \UCitem{Productos de salida} { %Productos de salida
    \begin{UClist}
		\UCli	Actualización de los libros disponibles.
    \end{UClist}
  }

  \UCitem{Cliente} { %Cliente
    \cdtRef{Actor:Bibliotecario}{Bibliotecario}
  }

  \UCitem{Mecanismo de medición} { %Mecanismo de medición
    \begin{UClist}
      \UCli Respuesta inmediata
    \end{UClist}
  }
  \UCitem{Interrelación con otros procesos} { %Interrelación con otros procesos
  }


\end{Proceso}
%========================================================
%Descripción de tareas
%-----------------------------------------------
\begin{PDescripcion}

  \Ppaso Bibliotecario
	\begin{enumerate}
		\Ppaso[\itarea] \cdtLabelTask{T1-P2.1:Bibliotecario}{Recepción de Presupuesto} El \cdtRef{Actor:Bibliotecario}{Bibliotecario} recibe el comunicado de Recursos Materiales sobre el presupuesto para comprar libros e informa a los jefes de académia.
	\end{enumerate}

	\begin{enumerate}
		\Ppaso[\itarea] \cdtLabelTask{T2-P2.1:Bibliotecario}{Recepción de petición de Libros} El \cdtRef{Actor:Bibliotecario}{Bibliotecario} recibe la lista con la petición de libros a pedir y junta en una sola lista.
	\end{enumerate}

	\begin{enumerate}
		\Ppaso[\itarea] \cdtLabelTask{T3-P2.1:Bibliotecario}{Recepción de Libros} El \cdtRef{Actor:Bibliotecario}{Bibliotecario} recibe los ejemplares comprados por Recursos Materiales y verifica la entrega para proceder a la prepraración de los libros.
	\end{enumerate}	
%%		\cdtRefTask{T2-P4.3:Bibliotecario}{Autentificar Lector.}	
\end{PDescripcion}


%------------------------------------------------------------------------------------------------------------------------------------------------


