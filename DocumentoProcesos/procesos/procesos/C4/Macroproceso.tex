

%------------------------------------------------------------------------------------------------------------------------------------------------


%========================================================
%Proceso Cambiar estado de Alumno
%========================================================

%========================================================
% Descripción general del proceso
%-----------------------------------------------
\begin{Proceso}{P4.5}{Macro Proceso PGU} {
  
  %-------------------------------------------
  %Resumen

  En este apartado se mostrará el macroproceso de la gestión de usuarios y como funcionan los subprocesos en conjunto.
  
  
En el macroproceso se indica que el Lector podrá escoger cual de las 4 subprocesos quiere realizar ya sea \cdtIdRef{P4.1}{Dar de Alta Lector}, \cdtIdRef{P4.2}{Dar de Baja Lector}, \cdtIdRef{P4.3}{Actulizar datos de Lector} ó \cdtIdRef{P4.5}{Consultar historial de Lector} y dependiendo de su selección se seguirán los pasos correspondientes da cada subproceso.  


  %-------------------------------------------
  %Diagrama del proceso

  \noindent La Figura \cdtRefImg{P4.5}{Macro Proceso PGU} muestra las actividades que se realizan para llevar a cabo el proceso descrito anteriormente.

  \Pfig[0.95]{./procesos/C4/Images/GU4_5-Macro.png}{P4.5}{Macro Proceso PGU}

} {P4.5:Macro Proceso PGU}

  %-------------------------------------------
  %Elementos del proceso

  \UCitem{Actores} { %Actores
    \cdtRef{Actor:Bibliotecario}{Bibliotecario} y \cdtRef{Actor:Lector}{Lector}.
  }

  \UCitem{Objetivo} { %Objetivo
    Validar Datos y generar el registro del \cdtRef{Actor:Lector}, para que este tenga privilegios según el perfil en la biblioteca.
  }

  \UCitem{Insumos de entrada} { %Insumos de entrada
  	\begin{UClist}
  		\UCli Datos del Formulario \cdtIdRef{F4.1}{Dar de alta Lector}.
    \end {UClist}
  }
  
  \UCitem{Proveedores} { %Proveedores
    \cdtRef{Actor:Lector}{Lector}
  }

  \UCitem{Productos de salida} { %Productos de salida
    \begin{UClist}
    
      \UCli Tabla generada con los datos correspondientes del  \cdtRef{Actor:Lector}{Lector} y su Estado en el sistema.
      \UCli   Notificación \cdtIdRef{MSJ4.5}{Operación exitosa}.
    \end{UClist}
  }

  \UCitem{Cliente} { %Cliente
    \cdtRef{Actor:Lector}{Lector}
  }

  \UCitem{Mecanismo de medición} { %Mecanismo de medición
    \begin{UClist}
      \UCli Respuesta inmediata
      
    \end{UClist}
  }
  \UCitem{Interrelación con otros procesos} { %Interrelación con otros procesos
  \begin{UClist}
	      \UCli	\cdtIdRef{P4.1}{Dar de Alta Lector}    
	      \UCli	\cdtIdRef{P4.2}{Dar de Baja Lector}
	      \UCli	\cdtIdRef{P4.3}{Actualizar Datos de Lector}
         \UCli \cdtIdRef{P4.4}{Consultar historial de Lector}
        \end{UClist}
  }


\end{Proceso}

%========================================================
%Descripción de tareas
%-----------------------------------------------
\begin{PDescripcion}

  %Actor: Aspirante
  \Ppaso Lector

    \begin{enumerate}

      %Tarea a
\Ppaso[\itarea] \cdtLabelTask{T1-P4.5:Lector}{Consultar Historial.} El \cdtRef{Actor:Lector}{Lector} al haber iniciar sesión, seguirá el proceso especificado en el subproceso \cdtIdRef{P4.4}{Consultar historial de Lector}

	

    \end{enumerate}
    
    
      %Actor: SAEV2.0
  \Ppaso Bibliotecario 

    \begin{enumerate}

      %Tarea a
      %Tarea a
      \Ppaso[\itarea] \cdtLabelTask{T1-P4.5:Bibliotecario}{Dar de Alta Lector.} Aqui se seguirán los pasos correspondientes del subproceso \cdtIdRef{P4.1}{Dar de Alta Lector} si es que el lector eligió dicha opción.

%referenciar tareas \cdtRefTask{T2-P0.1:SAEV2.0}{Notifica solicitud fuera de periodo de registro.}


      %Tarea b
      \Ppaso[\itarea] \cdtLabelTask{T2-P4.5:Bibliotecario}{Dar de Baja Lector.} Aqui se seguirán los pasos correspondientes del subproceso \cdtIdRef{P4.2}{Dar de Baja Lector} si es que el lector eligió dicha opción.
      
      %Tarea c
      \Ppaso[\itarea] \cdtLabelTask{T3-P4.5:Bibliotecario}{Actualizar datos de Lector.} Aqui se seguirán los pasos correspondientes del subproceso \cdtIdRef{P4.3}{Actualizar datos de Lector} si es que el lector eligió dicha opción.     

          \end{enumerate}

\end{PDescripcion}


%------------------------------------------------------------------------------------------------------------------------------------------------
