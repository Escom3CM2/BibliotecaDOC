%========================================================
%Proceso
%========================================================

%========================================================
% Descripción general del proceso
%-----------------------------------------------
\begin{Proceso}{P5.2}{Solicitud de renovación de credencial} {
  
  %-------------------------------------------
  %Resumen

  Proceso que realiza el \cdtRef{Actor:Estudiante}{Estudiante} para renovar los datos de su credencial, y así poder tener
  actualizados sus datos, lo que evitará almacenar datos erróneos.
  Si el periodo de renovación se encuentra vigente, el sistema permite al \cdtRef{Actor:Estudiante}{Estudiante} ingresar la
  información necesaria para obtener la información actualizada del alumno, de lo contrario le
  notifica que se encuentra fuera de dicho periodo y por lo tanto no se pueden renovar sus datos en
  el sistema. Cuando se actualizan los datos, el sistema verifica si la información proporcionada es
  correcta, de no ser el caso, se notifica que existen errores en la información para que se realicen las
  correcciones pertinentes antes de su envío.
  Una vez que la información es enviada correctamente, el Sistema debe asegurarse de que el
  \cdtRef{Actor:Estudiante}{Estudiante} ya se encuentra previamente registrado como usuario de la biblioteca con cuenta activa,
  en caso de que no esté activa la cuenta, el sistema notifica que la solicitud no procede.

  %-------------------------------------------
  %Diagrama del proceso

  \noindent La Figura \cdtRefImg{P5.2}{Renovación de Credencial} muestra las actividades que se realizan para llevar a cabo el proceso descrito anteriormente.

  \Pfig[0.95]{./procesos/C5_Credenciales/Images/PA5_2-RenovaciondeCredencial.png}{P5.2}{Renovación de Credencial}

} {P5.2:Cuenta}

  %-------------------------------------------
  %Elementos del proceso

  \UCitem{Actores} { %Actores
    \cdtRef{Actor:Estudiante}{Estudiante} y \cdtRef{Actor:SistemaBibliotecario}{Sistema Bibliotecario}
  }

  \UCitem{Objetivo} { %Objetivo
    Actualizar los datos del usuario de la biblioteca
  }

  \UCitem{Insumos de entrada} { %Insumos de entrada
    \begin{UClist}
      \UCli Credencial
      \UCli Horario de clases
      \UCli Datos a actualizar
    \end {UClist}
  }
  
  \UCitem{Proveedores} { %Proveedores
    \cdtRef{Actor:Estudiante}{Estudiante}
  }

  \UCitem{Productos de salida} { %Productos de salida
    Credencial actualizada
  }

  \UCitem{Cliente} { %Cliente
    \cdtRef{Actor:Estudiante}{Estudiante}
  }

  \UCitem{Mecanismo de medición} { %Mecanismo de medición
    Ninguno
  }
  \UCitem{Interrelación con otros procesos} { %Interrelación con otros procesos
    Gestión de usuarios
  }
  \UCitem{Tipo de solicitud al que aplica el proceso} { %Tipo de solicitud al que aplica el proceso
    Usuarios de la biblioteca
  }


\end{Proceso}

%========================================================
%Descripción de tareas
%-----------------------------------------------
\begin{PDescripcion}

  %Actor: Alumno
  \Ppaso Alumno

    \begin{enumerate}

      %Tarea a
      \Ppaso[\itarea] \cdtLabelTask{T1-P5.2:Alumno.}{Solicita renovación de crendencia.}  Para solicitar una renovación de credencial, ésta se deberá hacer presentando la credencial de la biblioteca y el horario de clases, así como la información a renovar quedando a la espera de alguno de los siguientes mensajes:

      %Eventos
      \begin{itemize}
        %Evento 1
        \item Fuera de días de trámite.
        %Evento 2
        \item Fuera de horario de trámite.
        %Evento 3
        \item Fuera de meses de trámite.
      \end{itemize}

      Si no se envía ninguno de los mensajes previos, el usuario tendrá renovados los datos de su credencial.

    \end{enumerate}

  %Actor: Sistema
  \Ppaso Sistema

    \begin{enumerate}

      %Tarea a
      \Ppaso[\itarea] \cdtLabelTask{T1-P5.2:Sistema.}{Renovación de credencial.} El \cdtRef{Actor:SistemaBibliotecario}{Sistema Bibliotecario} valida que no haya errores en la información. Si la información es correcta, el \cdtRef{Actor:SistemaBibliotecario}{Sistema Bibliotecario} obtendrá la siguiente información:

      %Eventos
      \begin{itemize}
        %Evento 1
        \item Días de trámite.
        %Evento 2
        \item Horarios de trámite.
        %Evento 3
        \item Meses de trámite.
      \end{itemize}

      Si el usuario solicita la renovación en los días y horarios permitidos, el \cdtRef{Actor:SistemaBibliotecario}{Sistema Bibliotecario} renovará los datos del usuario y le enviará el mensaje “Credencial lista”, en caso contrario enviará alguno de los siguientes mensajes (dependiendo del caso):

      %Eventos
      \begin{itemize}
        %Evento 1
        \item Fuera de días de trámite.
        %Evento 2
        \item Fuera de horario de trámite.
        %Evento 3
        \item Fuera de meses de trámite.
      \end{itemize}

    \end{enumerate}

\end{PDescripcion}