%========================================================
%Proceso
%========================================================

%========================================================
% Descripción general del proceso
%-----------------------------------------------
\begin{Proceso}{P2.2}{Cotejar Existencia de Material} {
  
  %-------------------------------------------
  %Resumen

  Proceso que realiza el  encargado de \cdtRef{Actor:Procesos_T}{Procesos Técnicos} cada año con la finalidad de conocer el comparativo de los libros físicos, material audiovisual y/o Equipo de Cómputo que se tienen en biblioteca con los materiales que debería haber de acuerdoa la última entrada de ejemplares "último estado de la base de datos".
  
  El encargado de \cdtRef{Actor:Procesos_T}{Procesos Técnicos} procede a cerrar la biblioteca para limpiar los materiales que desee inventariar, estos pueden ser material audiovisual, libros o equipo de cómputo y revisar su estado físico,para después obtener su código e introducirlo en el sistema donde se crea un formato de inventario con los códigos.  Éste formato se envía al sistema interno del IPN donde se realiza el comparativo y se devuelve el oficio con las diferencias de inventario.
  
  Se identifican los materiales que están reportados como faltantes, si éstos materiales son libros o material audiovisual, se verifica si un alumno posee el material, de ser así se agrega como parte del inventario y en caso contrario se realiza una búsqueda extenuante, si después de la búsqueda no se encuentra el material, se cambia el estado del material a extraviado.
En el supuesto cado de ser equipo de cómputo se vuelve a buscar en la biblioteca, de no ser encontrado se actualiza el estado a extraviado.
  
  %-------------------------------------------
  %Diagrama del proceso
  \noindent La Figura \cdtRefImg{P2.2}{Cotejar Existencia de Material} muestra las actividades que se realizan para llevar a cabo el proceso descrito anteriormente.

  \Pfig[1.0]{./procesos/C2/Images/cotejarExistencias.png}{P2.2}{Cotejar Existencia de Material}

} {P2.2:Cotejar Existencia de Material}

  %-------------------------------------------
  %Elementos del proceso
  \UCitem{Actores} { %Actores
	\cdtRef{Actor:Procesos_T}{Procesos Técnicos}
	\cdtRef{Actor:Servidor_P}{Servidor IPN}
  }

  \UCitem{Objetivo} { %Objetivo
    Comparar la existencia de material en biblioteca con el último estado de la base de datos.
  }

  \UCitem{Insumos de entrada} { %Insumos de entrada
  	\begin{UClist}
  		\UCli Identificadores de material proporcionados por el \cdtRef{Actor:Procesos_T}{Procesos Técnicos} al ser escaneados:
		\UCli \cdtIdRef{F2.5}{Escanear Libros}.
		\UCli \cdtIdRef{F2.6}{Escanear Material Audiovisual}.
		\UCli \cdtIdRef{F2.7}{Escanear Equipo de Cómputo}. 
				  		 
  		 
     	
    \end {UClist}
  }
  
  \UCitem{Proveedores} { %Proveedores
    \cdtRef{Actor:Procesos_T}{Procesos Técnicos}
  }

  \UCitem{Productos de salida} { %Productos de salida
    \begin{UClist}
		\UCli	Documentos generados de inventario.
		\UCli \cdtIdRef{D2.1}{Libros Escaneados}
		\UCli \cdtIdRef{D2.2}{Material Audiovisual Escaneado}.
		\UCli \cdtIdRef{D2.3}{Equipo de Cómputo Escaneado}
    \end{UClist}
  }

  \UCitem{Cliente} { %Cliente
    \cdtRef{Actor:Procesos_T}{Procesos Técnicos}
  }

  \UCitem{Mecanismo de medición} { %Mecanismo de medición
    \begin{UClist}
      \UCli Respuesta inmediata
    \end{UClist}
  }
  \UCitem{Interrelación con otros procesos} { %Interrelación con otros procesos
  	    \begin{UClist}
  	    	\UCli Préstamo de Material.
  	    \end{UClist}
  }


\end{Proceso}
%========================================================
%Descripción de tareas
%-----------------------------------------------
\begin{PDescripcion}

  \Ppaso Procesos Técnicos
	\begin{enumerate}
		\Ppaso[\itarea] \cdtLabelTask{T1-P2.2:Procesos_T}{Escanear Material} El encargado de \cdtRef{Actor:Procesos_T}{Procesos Técnicos} cierra la biblioteca y comienza la limpieza del material que desea escanear, posteriormente introduce uno a uno los identificadores de los materiales a escanear en el sistema.
		
		\Ppaso[\itarea] \cdtLabelTask{T2-P2.2:Procesos_T}{Envío y Recepción de Comparativo} El encargado de \cdtRef{Actor:Procesos_T}{Procesos Técnicos}  envía al servidor interno del IPN el documento de los materiales escaneados: \begin{itemize}
		 \item \cdtIdRef{D2.1}{Libros Escaneados}
		 \item  \cdtIdRef{D2.2}{Material Audiovisual Escaneado}.
		\item  \cdtIdRef{D2.3}{Equipo de Cómputo Escaneado}
		\end{itemize} Recibe como respuesta \begin{itemize}
		\item \cdtIdRef{D2.4}{Libros Faltantes}
		\item \cdtIdRef{D2.5}{Material Audiovisual Faltantes}
		\item \cdtIdRef{D2.6}{Equipo de Cómputo Faltante}
		\end{itemize}
		
		\Ppaso[\itarea] \cdtLabelTask{T3-P2.3:Procesos_T}{Identificación de Material Faltanteq} El encargado de \cdtRef{Actor:Procesos_T}{Procesos Técnicos} consulta los materiales faltantes directo de la base. Procede a verificar cual fue el desenlace del material, en el peor de los casos tendra que cambiar su estado en la base como extraviado.
	\end{enumerate}
	
		\Ppaso Servidor IPN
		\begin{enumerate}
			\Ppaso[\itarea] \cdtLabelTask{T1-P2.4:Servidor_I}{Procesar Formatos:} Se encarga de procesar los documentos:\begin{itemize}
			\item \cdtIdRef{D2.1}{Libros Escaneados}
			\item \cdtIdRef{D2.2}{Material Audiovisual Escaneado}.
			\item \cdtIdRef{D2.3}{Equipo de Cómputo Escaneado}
			\end{itemize} para despues responder con los documentos:
			\begin{itemize}
		\item \cdtIdRef{D2.4}{Libros Faltantes}
		\item \cdtIdRef{D2.5}{Material Audiovisual Faltantes}
		\item \cdtIdRef{D2.6}{Equipo de Cómputo Faltante}
		\end{itemize}
			
		\end{enumerate}
\end{PDescripcion}