%=============================================
% Descripción de actores

\chapter{Actores del sistema}
\label{chapter:ActoresDelSistema}

En el presente capítulo se definen los actores que participan en el sistema.

%============== Jefe de biblioteca=========================================
\cdtLabel{Actor:Jefe de Biblioteca}{}
\begin{Actor}{Jefe de Biblioteca}{Persona que puede ser un docente que hace uso del sistema para gestionar al personal.}
	\item[Área:] Biblioteca
	\item[Responsabilidades:] \hspace{1pt}
	\begin{itemize}
		\item Dar de alta a nuevo personal.
		\item Consultar personal administradora del sistema.
		\item Modificar datos del personal.
		\item Dar de baja al personal.
	\end{itemize}
\item[Perfil:] \hspace{1pt}
	\begin{itemize}
		\item Debe ser una persona perteneciente al instituto.
		\item Debe contar facilidad de palabras.		
		\item Debe ser responsable.
		\item Ser asignado por la Subdireccion de Servicios Educativos e Integracion Social.
	\end{itemize}
\end{Actor}

%==============Alumno=========================================
\cdtLabel{Actor:Lector}{}
\begin{Actor}{Lector}{Persona que puede ser un alumno interno, externo o docente que hace uso de los servicios que la Biblioteca ofrece.}
	\item[Área:] No aplica
	\item[Responsabilidades:] \hspace{1pt}
	\begin{itemize}
		\item Proporcionar datos de identificación
		\item Consultar perfil
		\item Pagar Multas
		\item Devolver Material
	\end{itemize}
	\item[Perfil:] \hspace{1pt}
	\begin{itemize}
		\item Debe ser una persona perteneciente al instituto. 		
		\item Debe ser responsable.
		\item Debe ser cumplido.
	\end{itemize}
\end{Actor}

%==============  Bibliotecario ========================
\cdtLabel{Actor:Bibliotecario}{}
\begin{Actor}{Bibliotecario}{Persona encargada de gestionar el proceso interno de la gestión de usuarios. }
	\item[Área:] ---
	\item[Responsabilidades:] \hspace{1pt}
	\begin{itemize}
		\item Dar de Alta lector
		\item Dar de baja Lector
		\item Modificar datos del Lector
			\end{itemize}
\item[Perfil:] \hspace{1pt}
	\begin{itemize}
	\item Debe ser una persona perteneciente al instituto.
		\item Debe contar facilidad de palabras
		\item Capaz de trabajar trabajar bajo presión 
		\item Debe ser egresado o estar cursando ultimo año de nivel superior 		
		\item Debe ser responsable.
	\end{itemize}
\end{Actor}




%==============  PGU ========================
\cdtLabel{Actor:PGU}{}
\begin{Actor}{PGU}{Sistema de administración bibliotecario encargado de automatizar todo lo relacionado con la gestion de usuarios.}
	\item[Área:] ---
	\item[Responsabilidades:] \hspace{1pt}
	\begin{itemize}
		\item Verificar existencia de Lectores
		\item Verificar existencia de historiales de los lectores
		\item Generar tablas con la información asociada a los lectores
			\end{itemize}
\end{Actor}


%==============Procesos Técnicos=========================================
\cdtLabel{Actor:Procesos_T}{}
\begin{Actor}{Procesos Técnicos}{}
	\item[Área:] No aplica
	\item[Responsabilidades:] \hspace{1pt}
	\begin{itemize}
		\item Comunicar a los encargados de departamentos Académicos el presupuesto que ha sido asignado por academia.
		\item Reunir y agrupar información proviniente de cada una de las academias dónde se especifica que libros y/o material aaudiovisual junto con sus cantidades que deberán ser comprados.
		\item Enviar a recursos materiales el oficio que informa los elementos y cantidades que han sido requeridos por cada una de las academias. 
		\item Desempaquetar los productos recibidos y verificar que hayan sido recibidos la misma cantidad y tipo de productos que fueron enviados previamente de recursos materiales
		\item Separar los productos por tipo para después darlos de alta en el sistema.
		\item Dar de alta los productos en el sistema y etiquetarlos con su respectivo identificador.
		\item Poner sello a los productos y forrar en caso de ser necesario.
		\item Poner los productos que se acaban de dar de alta en el estante adecuado.
		\item Limpiar y Revisar que los libros, material audiovisual y equipo de cómputo se encuentren en buenas condiciones.
		\item Escanear código de barras de elementos dados de alta en el sistema.
		\item Enviar oficio al servidor del Politécnico con la información registrada al escanear los productos.
		\item Revisar el oficio recibido de parte del servidor del Politécnico para verificar aquellas anormalidades en el inventario
	\end{itemize}
	\item[Perfil:] \hspace{1pt}
	\begin{itemize}
		\item Haber sido contratado por la jefa del área de Biblioteca.
		\item Haber tenido capacitación para poder realizar etiquetado y sellado de libros.
		\item Tener conocimiento de los procesos implicados a su cargo.
		\item Tener conocimientos básicos de computación.
		\item Debe ser responsable.	
		\item Revisar el estado del libro, material audiovisual y equipo de cómputo.	
	\end{itemize}
\end{Actor}

%==============Jefe de Academia=========================================
\cdtLabel{Actor:Jefe_A}{}
\begin{Actor}{Jefe de Academia}{}
	\item[Área:] No aplica
	\item[Responsabilidades:] \hspace{1pt}
	\begin{itemize}
		\item Evaluar el presupuesto aceptado para comprar los libros y/o material audiovisual pertinentes.
		\item Informar a los maestros para que realicen envío de propuestas de libros y/o material audiovisual a ser considerados como posibles compras.
		\item Recibir y Evaluar las propuestas de los maestros seleccionando los elementos de mayor importancia de acuerdo al presupuesto permitido, después seleccionar las cantidades para las cuales ha sido aprobado el presupuesto.
		\item Enviar oficio a Procesos Técnicos con los elementos aprobados y sus cantidades respectivas verificando que no excedan del presupuesto asignado.
	\end{itemize}
	\item[Perfil:] \hspace{1pt}
	\begin{itemize}
		\item Haber sido aceptado como jefe de academia.
	\end{itemize}
\end{Actor}

%==============Recursos Materiales=========================================
\cdtLabel{Actor:Recursos_M}{}
\begin{Actor}{Recursos Materiales}{}
	\item[Área:] No aplica
	\item[Responsabilidades:] \hspace{1pt}
	\begin{itemize}
		\item Recibir el oficio de libros y cantidades de los mismos a comprar.
		\item Cotizar con los proveedores el mejor precio posible al momento de comprar los libros.
		\item Hacer el pago de libros.
		\item Acordar fecha y hora de entrega de los libros.
		\item Enviar los libros recien adquiridso al encargado de procesos técnicos junto con su recibo de compra, resaltando las cantidades por cada libro que fueron adquiridos.
	\end{itemize}
	\item[Perfil:] \hspace{1pt}
	\begin{itemize}
		\item Haber sido aceptado como jefe de Recursos Materiales.
	\end{itemize}
\end{Actor}

%==============Maestro=========================================
\cdtLabel{Actor:Maestro}{}
\begin{Actor}{Maestro}{}
	\item[Área:] No aplica
	\item[Responsabilidades:] \hspace{1pt}
	\begin{itemize}
		\item Enviar sugerencias de libros a ser comprados al jefe de academia, considerando los pertinentes segun su experiencia y necesidades.
	\end{itemize}
	\item[Perfil:] \hspace{1pt}
	\begin{itemize}
		\item Tener la plaza de maestro en la institución.
	\end{itemize}
\end{Actor}

%===============Proveedor========================================
\cdtLabel{Actor:Proveedor}{}
\begin{Actor}{Proveedor}{}
	\item[Área:] No aplica
	\item[Responsabilidades:] \hspace{1pt}
	\begin{itemize}
		\item Enviar cotizaciones de los libros a surtir.
		\item Generar Factura de cobro por el monto total de los libros a ser comprados.
		\item Generar comprobante de pago una vez que se efectuó la compra.
		\item Enviar los libros o material que haya sido comprado dentro de la fecha y horario acordado al momento de la compra.
	\end{itemize}
	\item[Perfil:] \hspace{1pt}
	\begin{itemize}
		\item Tener el permiso legal para  vender los libros.
		\item Manejar precios adecuados para realizar el trato.
	\end{itemize}
\end{Actor}

%===============Servidor Politécnico========================================
\cdtLabel{Actor:Servidor_P}{}
\begin{Actor}{Servidor del IPN}{}
	\item[Área:] No aplica
	\item[Responsabilidades:] \hspace{1pt}
	\begin{itemize}
		\item Procesar oficio de control de inventario y destacar anormalidades en este, es decir faltante de algún producto.
	\end{itemize}
	\item[Perfil:] \hspace{1pt}
	\begin{itemize}
		\item --
	\end{itemize}
\end{Actor}





