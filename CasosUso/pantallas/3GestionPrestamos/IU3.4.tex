\newpage
\subsection{IU3.4 Registrar Prestamo de Material}

\subsubsection{Objetivo}
	Pantalla de Registro de Material donde el bibliotecario podra editar las observaciones del material si asi lo requiere, removerlo en caso de que se haya equivocado al introducir el ID del material y lo mas importante registrar el prestamo que se esta haciendo a un lector interno.  

\subsubsection{Diseño}
	Tenemos los botones: \IUbutton{Registrar}, \IUbutton{Modificar Observaciones}, \IUbutton{Remover} y \IUbutton{Regresar} y una lista desplegable donde seleccionara el tipo de material, por ultimo la tabla donde iran apareciendo los materiales que se quieren prestar y donde apareceran sus datos de estos.  \\\\

\IUfig[.7]{3GestionPrestamos/GUI3_4}{IU3.4}{PrestamoInterno}

\subsubsection{Salidas}
	\begin{Citemize}
		\item Actualizacion de la BD 
	\end{Citemize}
	
\subsubsection{Entradas}
	\begin{Citemize}
		\item El ID del material que se quiere prestar
	\end{Citemize}

\newpage
\subsection{IU3.4a Prestamo Externo}

\subsubsection{Objetivo}
	Pantalla donde se visualiza los datos que se requieren para poder prestar un libro a un lector externo 

\subsubsection{Diseño}
	Esta pantalla contiene los campos: Folio y ID de material que son los necesarios para que el bibliotecario pueda hacer el proceso de prestamo a un lector externo.  \\\\

\IUfig[.7]{3GestionPrestamos/GUI3_4a}{IU3.4a}{PrestamoExterno}

\subsubsection{Salidas}
	\begin{Citemize}
		\item Actualizacion de la BD 
	\end{Citemize}
	
\subsubsection{Entradas}
	\begin{Citemize}
		\item El Folio del formato fisico para el prestamo interbibliotecario
		\item El ID del material para identificar que libro se esta prestando
	\end{Citemize}