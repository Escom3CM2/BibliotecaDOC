\newpage
\subsection{IU3.2 Inicio Interno}

\subsubsection{Objetivo}
	Pantalla donde se visualiza el inicio para acceder a los procesos que se pueden manejar con los Lectores Externos. 

\subsubsection{Diseño}
	Esta pantalla contiene el boton \IUbutton{Verificar}, se muestran los campos: Nombre del Alumno, Boleta del Alumno y una lista Desplegable con las Escuelas Integradas al IPN

\IUfig[.7]{3GestionPrestamos/GUI3_2}{IU3.2}{InicioInterno}

\subsubsection{Salidas}
	\begin{Citemize}
		\item Muestra el \MSGref{MSJ3.1}{Error al conectar a la BD}
		\item Muestra el \MSGref{MSJ3.2}{Lector Inexistente}
		\item Muestra el \MSGref{MSJ3.21}{Credencial no vigente}
		\item Muestra el \MSGref{MSJ3.14}{Limite de prestamos alcanzado}
		\item Redireccionamiento a \IUref{IU3.2.1}{Lector Interno Apto}
	\end{Citemize}
	
\subsubsection{Entradas}
	\begin{Citemize}
		\item El bibliotecario ingresara los campos: Id del Lector
	\end{Citemize}
	
\newpage
\subsection{IU3.2.1 Lector Interno Apto}

\subsubsection{Objetivo}
	Pantalla donde se visualiza el Menu de los procesos que el lector interno puede realizar. 

\subsubsection{Diseño}
	Esta pantalla contiene los botones \IUbutton{Verificar}, \IUbutton{Prestamos}, \IUbutton{Devoluciones}, \IUbutton{Prestamo Interbibliotecario}, se muestran los campos: ID del Usuario en cuestion  \\\\

\IUfig[.7]{3GestionPrestamos/GUI3_2a}{IU3.2.1}{LectorInternoApto}

\subsubsection{Salidas}
	\begin{Citemize}
		\item El \MSGref{MSG3.16}{Lector Apto}
	\end{Citemize}
	
\subsubsection{Entradas}
	\begin{Citemize}
		\item El bibliotecario dara clic en los botones \IUbutton{Prestamos},\IUbutton{Devoluciones},\IUbutton{Prestamo Interbibliotecario}
	\end{Citemize}

\newpage
\subsection{IU3.2.2 Lector con Devoluciones Pendientes}

\subsubsection{Objetivo}
	Pantalla donde se visualiza una tabla con todos los prestamos asociados al lector interno y los prestamos con Devoluciones tardias sombreados de color rojo. 

\subsubsection{Diseño}
	Esta pantalla contiene los botones \IUbutton{Verificar}, \IUbutton{Generar Multa}, una tabla con los prestamos por devolver asociados al Lector y donde las Devoluciones con Estado Tardo se sombrearan en rojo, una columna donde podemos seleccionar las devoluciones que se haran y un monto Parcial de la multa que se ha generado por Devolucion Tarda \\\\

\IUfig[.7]{3GestionPrestamos/GUI3_2b}{IU3.2.2}{LectorDevolucionesPendientes}

\subsubsection{Salidas}
	\begin{Citemize}
		\item El \MSGref{MSJ3.6}{Lector con Devoluciones Atrasadas}
		\item Redireccion al CU Generar Multa
	\end{Citemize}
	
\subsubsection{Entradas}
	\begin{Citemize}
		\item Seleccion de la columna Devolver, para seleccionar los materiales que se estan regresando
		\item El bibliotecario dara clic en el boton \IUbutton{Generar Multa}
	\end{Citemize}

\newpage
\subsection{IU3.2.3 Lector con Multas Pendientes}

\subsubsection{Objetivo}
	Pantalla donde se visualiza una tabla con todos las multas asociadas al lector interno que no hayan pagado y los datos referentes a estas y realizar las operaciones adecuadas a estas.

\subsubsection{Diseño}
	Esta pantalla contiene los botones \IUbutton{Imprimir}, \IUbutton{Cancelar Multa}, \IUbutton{Registrar Multa}, una tabla con las multas que no se hayan pagado asociadas al lector, una columna donde podemos seleccionar las multas con las que se quiere realizar alguna operacion \\\\

\IUfig[.7]{3GestionPrestamos/GUI3_2c}{IU3.2.3}{LectorMultasPendientes}

\subsubsection{Salidas}
	\begin{Citemize}
		\item El \MSGref{MSJ3.5}{Lector con Multas}
		\item El \MSGref{MSJ3.12}{Pago de Multa}
		\item El \MSGref{MSJ3.18}{Cancelacion de Multa Exitosa}
		\item El \MSGref{MSJ3.19}{Impresion de Multa}
		\item EL \MSGref{MSJ3.20}{Error Cancelacion de Multa}
	\end{Citemize}
	
\subsubsection{Entradas}
	\begin{Citemize}
		\item Seleccion de la columna Pago, para seleccionar las multas con la que se haga alguna operacion
	\end{Citemize}