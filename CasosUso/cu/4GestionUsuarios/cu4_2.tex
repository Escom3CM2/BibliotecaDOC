% \IUref{IUAdmSucur}{Bibliotecario}
% \IUref{IUConsDA}{Actualizar Lector}
% Version 1


% Copie este bloque por cada caso de uso:
%-------------------------------------- COMIENZA descripción del caso de uso.

%\begin{UseCase}[archivo de imágen]{UCX}{Nombre del Caso de uso}{
	\begin{UseCase}{CU4.2}{Actualizar Lector}{
		El bibliotecario puede actualizar a un lector(alumno,docente) ya existente en el sistema, con la finalidad de tener almacenada la información más reciente del lector que hace uso de la biblioteca para consulta y control de la misma. Los datos a actualizar del lector son: nombre, CURP, dirección, telefono, correo, contraseña, semestre o departamento.}
		\UCitem{Versión}{1.0}
		\UCitem{Estado}{En revisión}
		\UCitem{Actor}{Bibliotecario.}
		\UCitem{Propósito}{Almacenar en la base de datos la información más reciente de los lectores que hacen uso de la biblioteca.}
		\UCitem{Entradas}{Nombre, Primer apellido, Segundo apellido, CURP, dirección, teléfono, email, contraseña, semestre o departamento}
		\UCitem{Salidas}{
			
			\begin{itemize}
				\item Actualización del registro en la base de datos,
				\item Mensaje \MSGref{MSJ4.2}{Error. Datos incompletos. Favor de completar todos los campos de entrada.},
				\item Mensaje \MSGref{MSJ4.3}{Advertencia. Confirma que los datos ingresados sean correctos.},
				\item Mensaje \MSGref{MSJ4.5}{Error. Falló al conectarse a la BD.},
				\item Mensaje \MSGref{MSJ4.6}{Se actualizo exitosamente.}
			\end{itemize}
		}
		\UCitem{Precondiciones}{
			\begin{itemize}
				\item El lector debe de estar registrado en el sistema.
				\item El lector no debe tener multas.
				\item El lector no debe tener devoluciones pendientes.
				\item El lector debe de estar vigente en el instituto.
			\end{itemize}
		}
		\UCitem{Postcondiciones}{Lector actualizado en el sistema.}
		\UCitem{Tipo}{Caso de uso primario.}
		\UCitem{Autor}{Miguel Ángel Castañeda Sánchez.}
		\UCitem{Revisor}{---}
	\end{UseCase}
	
	
%%------------Trayectoria Principal-----------------%%		
	
	\begin{UCtrayectoria}{Principal}
	
		\UCpaso Despliega la \IUref{IU4.2}{Perfil del bibliotecario}.

		\UCpaso[\UCactor] Da clik en el botón \IUbutton{Gestión de usuarios}.

		\UCpaso Despliega el menú de gestión de usuarios. \IUref{IU4.4}{Menú de gestión de usuarios}

		\UCpaso[\UCactor] Da clik en el botón \IUbutton{Lista de Docentes} ó \IUbutton{Lista de estudiante}.\label{CU4.2ConectarBaseDatos}

		\UCpaso Se conecta a la base de datos. \Trayref{A}.

		\UCpaso Despliega la \IUref{IU4.6}{Listado de estudiantes} ó \IUref{IU4.8}{Lista de docentes}.

		\UCpaso[\UCactor] Elige el lector a modificar y da click en el \IUbutton{Actualizar}

		\UCpaso Despliega la \IUref{IU4.16}{Actualizar informacion del estudiante} ó \IUref{IU4.17}{Actualizar información del docente}.


		\UCpaso[\UCactor] Actualiza los datos que son: 
			\begin{itemize}
				\item Nombre.
				\item Primer apellido.
				\item Segundo apellido.
				\item CURP.
				\item Fecha de nacimiento.
				\item Dirección
				\item Teléfono
				\item Semestre o Departamento.
				\item Email.
				\item Contraseña
			\end{itemize} 
		\label{CU4.1FormularioLector}

		\UCpaso[\UCactor] Da clik en el botón \IUbutton{Actualizar}.

		\UCpaso Verifica los campos de los datos con base a la \BRref{RN4.1}{Campos no nulos.} \Trayref{B}.

		\UCpaso Válida que los datos ingresados cumplan los formatos con base a las Reglas de Negocio: 
				
			\begin{itemize}
				\item	\BRref{RN4.3}{Formato del nombre} 
				\item	\BRref{RN4.4}{Formato del CURP} 
				\item	\BRref{RN4.5}{Formato de la fecha} 
				\item	\BRref{RN4.6}{Formato de la dirección} 
				\item	\BRref{RN4.7}{Formato del teléfono} 
				\item	\BRref{RN4.8}{Formato del semestre} 
				\item	\BRref{RN4.9}{Formato del email}
				\item	\BRref{RN4.10}{Formato de la contrasseña}
				\item	\BRref{RN4.13}{Formato del departamento}
			\end{itemize} 
		\Trayref{C}.
		
	
		\UCpaso El sistema modifica el registro del lector.

		\UCpaso Muestra el Mensaje \MSGref{MSJ4.6}{Se actualizo exitósamente.}

		\UCpaso[\UCactor] Oprime el botón \IUbutton{Aceptar}.

		\UCpaso Redirige a la pantalla \IUref{UI4.6}{Listado de alumnos} ó pantalla \IUref{UI4.6}{Listado de docentes}.
		

	\end{UCtrayectoria}



%%------------Trayectorias Alternativas-----------------%%	


%%------------Trayectoria A-----------------%%
		\begin{UCtrayectoriaA}{A.}{Error al conectar a la base de datos}

			\UCpaso Muestra el Mensaje \MSGref{MSJ4.5}{Error. Falló al conectarse a la BD.}
			\UCpaso[\UCactor] Oprime el botón \IUbutton{Aceptar}.
			\UCpaso Continua en el paso \ref{CU4.2ConectarBaseDatos} del \UCref{CU4.1}.

		\end{UCtrayectoriaA}	

%%------------Trayectoria B-----------------%%
				
		\begin{UCtrayectoriaA}{B.}{El bibliotecario no ha llenado todos los campos}
			\UCpaso Muestra el Mensaje \MSGref{MSJ4.2}{Error. datos incompletos. Favor de completar todos los campos de entrada.}
			\UCpaso[\UCactor] Oprime el botón \IUbutton{Aceptar}.
			\UCpaso Continua en el paso \ref{CU4.1FormularioLector} del \UCref{CU4.1}.
			
		\end{UCtrayectoriaA}

%%------------Trayectoria C-----------------%%
		\begin{UCtrayectoriaA}{C.}{El Administrador ha ingresado los datos de manera incorrecta}

			\UCpaso Muestra el Mensaje \item Mensaje \MSGref{MSJ4.3}{Advertencia. Confirma que los datos ingresados sean correctos.},
			\UCpaso[\UCactor] Oprime el botón \IUbutton{Aceptar}.
			\UCpaso Continua en el paso \ref{CU4.1FormularioLector} del \UCref{CU4.1}.

		\end{UCtrayectoriaA}

%-------------------------------------- TERMINA descripción del caso de uso.

