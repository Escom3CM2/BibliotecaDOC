% \IUref{IUAdmSucur}{Bibliotecario}
% \IUref{IUConsDA}{Consultar Lector}
% Version 1


% Copie este bloque por cada caso de uso:
%-------------------------------------- COMIENZA descripción del caso de uso.

%\begin{UseCase}[archivo de imágen]{UCX}{Nombre del Caso de uso}{
	\begin{UseCase}{CU4.4}{Consultar Lector}{
		El bibliotecario puede consultar informacion del lector(alumno,docente). Y así, conocer en que estado se encuentra el lector. }
		\UCitem{Versión}{1.0}
		\UCitem{Estado}{En revisión}
		\UCitem{Actor}{Bibliotecario.}
		\UCitem{Propósito}{El bibliotecario podrá visualizar información del lector referente a los prestamos interno y externos, devoluciones, multas.}
		\UCitem{Entradas}{}
		\UCitem{Salidas}{
			
			\begin{itemize}
				\item Mensaje \MSGref{MSJ4.1}{Se registro exitosamente.},
				\item Mensaje \MSGref{MSJ4.2}{Error. datos incompletos. Favor de completar todos los campos de entrada.},
				\item Mensaje \MSGref{MSJ4.5}{Error. Falló al conectarse a la BD.},
			\end{itemize}
		}
		\UCitem{Precondiciones}{
			\begin{itemize}
				\item El lector debe de estar registrado en el sistema.
			\end{itemize}
		}
		\UCitem{Postcondiciones}{Lector modifica su estado en el sistema.}
		\UCitem{Tipo}{Caso de uso primario.}
		\UCitem{Autor}{Miguel Ángel Castañeda Sánchez.}
		\UCitem{Revisor}{---}
	\end{UseCase}
	
	
%%------------Trayectoria Principal-----------------%%		
	
	\begin{UCtrayectoria}{Principal}
	
		\UCpaso Despliega la \IUref{IU4.2}{Perfil del bibliotecario}.

		\UCpaso[\UCactor] Da clik en el botón \IUbutton{Gestión de préstamos}.

		\UCpaso Despliega el menú de gestión de préstamos. \IUref{IU4.10}{Menú de gestión pŕestamos}

		\UCpaso[\UCactor] Da clik en el botón \IUbutton{Interno} ó \IUbutton{Externo}

		\UCpaso Despliega la \IUref{IU4.11}{Préstamo interno} ó \IUref{IU4.12}{Préstamo externo}.

		\UCpaso[\UCactor] Ingresa el dato en el campo de búsqueda para encontrar al lector. Los campos de búsqueda son:
			\begin{itemize}
				\item  Id del lector
				\item  Nombre
				\item  Número de boleta del IPN
				\item  Escuela
			\end{itemize}
		\label{CU4.1BuscarLector}

		\UCpaso[\UCactor] Da click en el \IUbutton{Verificar}\label{CU4.2ConectarBaseDatos}


		\UCpaso Verifica el ó los campos de los datos con base a la \BRref{RN4.1}{Campos no nulos.} \Trayref{A}.

		\UCpaso Válida el(los) campos de busqueda los formato con base a las reglas de negocio: 
				
			\begin{itemize}
				\item   \BRref{RN4.2}{Formato del número de boleta del IPN.}
				\item 	\BRref{RN4.3}{Formato del nombre.}
				\item	\BRref{RN4.24}{Formato del id del lector} 
				\item	\BRref{RN4.25}{Formato de la escuela} 
			\end{itemize} 
		\Trayref{B}.

		\UCpaso Se conecta a la base de datos. \Trayref{C}.

		\UCpaso Despliega la información del lector en la \IUref{IU4.11}{Préstamo interno} ó \IUref{IU4.12}{Préstamo externo}.

		\UCpaso[\UCactor] puede consultar:
			\begin{itemize}
				\item Menú de acción
					\begin{itemize}
         				\item Préstamo. \Trayref{D}
 				        \item Préstamo TT y audiovisuales. \Trayref{E}
 				        \item Devoluciones. \Trayref{F} 
 				        \item Préstamo interbibliotecario. \Trayref{G}
    				\end{itemize}
				\item Devoluciones atrasadas
					\begin{itemize}
						\item Devolver libros. \Trayref{H}
					\end{itemize}
				\item Tiene multas pendientes
					\begin{itemize}
						\item Consultar multas. \Trayref{I}
						\item Pagar multas. \Trayref{J}
					\end{itemize}
			\end{itemize}
		\label{CU4.3OperacionesLector}
	\end{UCtrayectoria}



%%------------Trayectorias Alternativas-----------------%%	

%%------------Trayectoria A-----------------%%
				
		\begin{UCtrayectoriaA}{A.}{El bibliotecario no ha llenado todos los campos}
			\UCpaso Muestra el Mensaje \MSGref{MSJ4.2}{Error. datos incompletos. Favor de completar todos los campos de entrada.}
			\UCpaso[\UCactor] Oprime el botón \IUbutton{Aceptar}.
			\UCpaso Continua en el paso \ref{CU4.1BuscarLector} del \UCref{CU4.4}.
			
		\end{UCtrayectoriaA}

%%------------Trayectoria B-----------------%%
		\begin{UCtrayectoriaA}{B.}{El Administrador ha ingresado los datos de manera incorrecta}

			\UCpaso Muestra el Mensaje \item Mensaje \MSGref{MSJ4.3}{Advertencia. Confirma que los datos ingresados sean correctos.},
			\UCpaso[\UCactor] Oprime el botón \IUbutton{Aceptar}.
			\UCpaso Continua en el paso \ref{CU4.1BuscarLector} del \UCref{CU4.4}.

		\end{UCtrayectoriaA}		

%%------------Trayectoria C-----------------%%
		\begin{UCtrayectoriaA}{C.}{Error al conectar a la base de datos}

			\UCpaso Muestra el Mensaje \MSGref{MSJ4.5}{Error. Falló al conectarse a la BD.}
			\UCpaso[\UCactor] Oprime el botón \IUbutton{Aceptar}.
			\UCpaso Continua en el paso \ref{CU4.2ConectarBaseDatos} del \UCref{CU4.4}.

		\end{UCtrayectoriaA}


%%------------Trayectoria D-----------------%%
		\begin{UCtrayectoriaA}{D.}{Préstamo}

			\UCpaso Inicia CU Préstamo
			\UCpaso Continua en el paso \ref{CU4.3OperacionesLector} del \UCref{CU4.4}.

		\end{UCtrayectoriaA}

%%------------Trayectoria E-----------------%%
		\begin{UCtrayectoriaA}{E.}{Préstamo TT y audiovisuales}

			\UCpaso Inicia CU Préstamo TT y audiovisuales
			\UCpaso Continua en el paso \ref{CU4.3OperacionesLector} del \UCref{CU4.4}.

		\end{UCtrayectoriaA}

%%------------Trayectoria F-----------------%%
		\begin{UCtrayectoriaA}{F.}{Devoluciones}

			\UCpaso Inicia CU Devoluciones
			\UCpaso Continua en el paso \ref{CU4.3OperacionesLector} del \UCref{CU4.4}.

		\end{UCtrayectoriaA}

%%------------Trayectoria G----------------%%
		\begin{UCtrayectoriaA}{G.}{Préstamo interbibliotecario}

			\UCpaso Inicia CU Préstamo interbibliotecario
			\UCpaso Continua en el paso \ref{CU4.3OperacionesLector} del \UCref{CU4.4}.

		\end{UCtrayectoriaA}

%%------------Trayectoria H-----------------%%
		\begin{UCtrayectoriaA}{H.}{Devolver libros}

			\UCpaso Inicia CU Devolver libros
			\UCpaso Continua en el paso \ref{CU4.3OperacionesLector} del \UCref{CU4.4}.

		\end{UCtrayectoriaA}				

%%------------Trayectoria I-----------------%%
		\begin{UCtrayectoriaA}{I.}{Consultar multas}

			\UCpaso Inicia CU Consultar multas
			\UCpaso Continua en el paso \ref{CU4.3OperacionesLector} del \UCref{CU4.4}.

		\end{UCtrayectoriaA}

%%------------Trayectoria J-----------------%%
		\begin{UCtrayectoriaA}{J.}{Pagar multas}

			\UCpaso Inicia CU Pagar multas
			\UCpaso Continua en el paso \ref{CU4.3OperacionesLector} del \UCref{CU4.4}.

		\end{UCtrayectoriaA}

%-------------------------------------- TERMINA descripción del caso de uso.

