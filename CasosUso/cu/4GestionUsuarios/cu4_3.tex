% \IUref{IUAdmSucur}{Bibliotecario}
% \IUref{IUConsDA}{Eliminar Lector}
% Version 1


% Copie este bloque por cada caso de uso:
%-------------------------------------- COMIENZA descripción del caso de uso.

%\begin{UseCase}[archivo de imágen]{UCX}{Nombre del Caso de uso}{
	\begin{UseCase}{CU4.3}{Eliminar Lector}{
		El bibliotecario puede elimnar al lector(alumno,docente) que no hagan uso del sistema.}
		\UCitem{Versión}{1.0}
		\UCitem{Estado}{En revisión}
		\UCitem{Actor}{Bibliotecario.}
		\UCitem{Propósito}{Los lectores que ya no usen el sistema pueden darse de baja del sistema bibliotecario.}
		\UCitem{Entradas}{Contraseña del bibliotecario}
		\UCitem{Salidas}{
			
			\begin{itemize}
				\item El registro del lector  se elimina de la base de datos,
				\item Mensaje \MSGref{MSJ4.5}{Error. Falló al conectarse a la BD.},
				\item Mensaje \MSGref{MSJ4.7}{Se elimino exitosamente.}
				\item Mensaje \MSGref{MSJ4.8}{Advertencia. El lector se eliminara del sistema.},
				\item Mensaje \MSGref{MSJ4.9}{Error. El lector tiene multas pendientes.},
				\item Mensaje \MSGref{MSJ4.10}{Error. El lector tiene material adeudado.},
				\item Mensaje \MSGref{MSJ4.11}{Advertencia. Constraseña incorrecta.},
			\end{itemize}
		}
		\UCitem{Precondiciones}{
			\begin{itemize}
				\item El lector debe de estar registrado en el sistema.
				\item El lector no debe tener multas.
				\item El lector no debe tener devoluciones pendientes.
			\end{itemize}
		}
		\UCitem{Postcondiciones}{Lector eliminado en el sistema.}
		\UCitem{Tipo}{Caso de uso primario.}
		\UCitem{Autor}{Miguel Ángel Castañeda Sánchez.}
		\UCitem{Revisor}{---}
	\end{UseCase}
	
	
%%------------Trayectoria Principal-----------------%%		
	
	\begin{UCtrayectoria}{Principal}
	
		\UCpaso Despliega la \IUref{IU4.2}{Perfil del bibliotecario}.

		\UCpaso[\UCactor] Da clik en el botón \IUbutton{Gestión de usuarios}.

		\UCpaso Despliega el menú de gestión de usuarios. \IUref{IU4.4}{Menú de gestión de usuarios}

		\UCpaso[\UCactor] Da clik en el botón \IUbutton{Lista de Docentes} ó \IUbutton{Lista de estudiante}.\label{CU4.2ConectarBaseDatos}

		\UCpaso Se conecta a la base de datos. \Trayref{A}.

		\UCpaso Despliega la \IUref{IU4.6}{Listado de estudiantes} ó \IUref{IU4.8}{Lista de docentes}.

		\UCpaso[\UCactor] Elige el lector a eliminar y da click en el \IUbutton{Eliminar}\label{CU4.1EliminarLector}

		\UCpaso Válida que el lector a eliminar cumpla las reglas de negocio: 
				
			\begin{itemize}
				\item	\BRref{RN4.20}{El lector no tiene multas} 
				\item	\BRref{RN4.21}{El lector no tiene adeudos} 

			\end{itemize} 
		\Trayref{B}.\Trayref{C}.


		\UCpaso Muestra el Mensaje \MSGref{MSJ4.8}{Advertencia. El lector se eliminara del sistema.}\label{CU4.3Contrasena}

		\UCpaso Solicita contraseña del bibliotecario.

		\UCpaso[\UCactor] Ingresa la contraseña y oprime el botón \IUbutton{Aceptar}.

		\UCpaso El sistema valida contraseña con base a la Regla de negocio \BRref{RN4.15}{Contraseña del bibliotecario} \Trayref{D}.

		\UCpaso El sistema modifica el registro del lector.

		\UCpaso Muestra el Mensaje \MSGref{MSJ4.6}{Se elimino exitósamente.}

		\UCpaso[\UCactor] Oprime el botón \IUbutton{Aceptar}.

		\UCpaso Redirige a la pantalla \IUref{UI4.6}{Listado de alumnos} ó pantalla \IUref{UI4.8}{Listado de docentes}.		
	\end{UCtrayectoria}



%%------------Trayectorias Alternativas-----------------%%	


%%------------Trayectoria A-----------------%%
		\begin{UCtrayectoriaA}{A.}{Error al conectar a la base de datos}

			\UCpaso Muestra el Mensaje \MSGref{MSJ4.5}{Error. Falló al conectarse a la BD.}
			\UCpaso[\UCactor] Oprime el botón \IUbutton{Aceptar}.
			\UCpaso Continua en el paso \ref{CU4.2ConectarBaseDatos} del \UCref{CU4.1}.

		\end{UCtrayectoriaA}	

%%------------Trayectoria B-----------------%%
				
		\begin{UCtrayectoriaA}{B.}{El lector tiene multas pendientes}
			\UCpaso Muestra el Mensaje \MSGref{MSJ4.9}{Error. El lector tiene multas. Favor de pagar su multa.}
			\UCpaso[\UCactor] Oprime el botón \IUbutton{Aceptar}.
			\UCpaso Continua en el paso \ref{CU4.1EliminarLector} del \UCref{CU4.1}.
			
		\end{UCtrayectoriaA}

%%------------Trayectoria C-----------------%%
				
		\begin{UCtrayectoriaA}{C.}{El lector tiene multas pendientes}
			\UCpaso Muestra el Mensaje \MSGref{MSJ4.10}{Error. El lector tiene material adeudado. Favor de entregar el material.}
			\UCpaso[\UCactor] Oprime el botón \IUbutton{Aceptar}.
			\UCpaso Continua en el paso \ref{CU4.1EliminarLector} del \UCref{CU4.1}.
			
		\end{UCtrayectoriaA}

%%------------Trayectoria D-----------------%%
		\begin{UCtrayectoriaA}{D.}{El bibliotecario ingreso su contraseña incorrectamente}

			\UCpaso Muestra el Mensaje \item Mensaje \MSGref{MSJ4.11}{Advertencia. Constraseña incorrecta.},
			\UCpaso[\UCactor] Oprime el botón \IUbutton{Aceptar}.
			\UCpaso Continua en el paso \ref{CU4.3Contrasena} del \UCref{CU4.1}.

		\end{UCtrayectoriaA}

%-------------------------------------- TERMINA descripción del caso de uso.

