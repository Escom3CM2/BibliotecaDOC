% Copie este bloque por cada caso de uso:
%-------------------------------------- COMIENZA descripción del caso de uso.

	\begin{UseCase}{CU2.3}{Enviar reporte de elementos escaneados}{
		Permite mostrar la información escaneada a partir del CU2.B1, la cual será usada para generar un reporte con los detalles de cada uno de los elementos contenidos en las listas escaneadas del sistema y así puedan ser enviadas.
	}
		\UCitem{Versión}{1}
		\UCitem{Actor}{Empleado de Procesos Técnicos}
		\UCitem{Propósito}{Generar y enviar un reporte con los elementos escaneados.} 

		\UCitem{Entradas}{
		Lista de elementos con los siguientes datos, de acuerdo con el caso que corresponda:
			\begin{itemize}
				\item ->LIBRO
				\item ISBN	
				\item Titulo	
				\item Autor(es)	
				\item Editorial	
				\item Fecha de Publicación	
				\item Numero de Páginas	
				\item Estado
				\item Precio	
			
				\item ->TT	
				\item Número de TT	
				\item Titulo	
				\item Autor(es)	
				\item Fecha de Presentación	
				\item Director(es)
			
				\item ->MATERIAL AUDIOVISUAL
				\item ID	
				\item Titulo	
				\item Autor(es)	
				\item Duracion	
				\item Fecha de Publicacion	
				\item Estado
				\item Precio
			
				\item ->EQUIPO DE CÓMPUTO	
				\item ID
				\item Marca
				\item Modelo
				\item Estado
				\item Precio
			\end{itemize}				
		}

		\UCitem{Salidas}{
			\begin{itemize}
				\item Mensajes para el usuario:
					\MSGref{MSJ2.4}{Envío de reporte exitoso}
					\MSGref{MSJ2.5}{Registros insuficientes.}
					\MSGref{MSJ2.6}{Reporte cancelado.}
					\MSGref{MSJ2.7}{Error de envío.}
					\MSGref{MSJ2.8}{Error de generación de reporte}
				\item Reporte de elementos escaneados.
			\end{itemize}				
		}

		\UCitem{Precondiciones}{ 
			\begin{itemize}
				\item Debe existir al menos un elemento escaneado en cada categoría para generar un reporte.
			\end{itemize}
		}
		\UCitem{Postcondiciones}{
			\begin{itemize}
			\item El remitente obtendrá una copia del reporte de manera digital.
			\end{itemize}				
		}
		\UCitem{Autor(es)}{
			\begin{itemize}
			\item Tejeda Martínez José Miguel
			\end{itemize}
		}
		\UCitem{Revisor(es)}{
			\begin{itemize}
				\item *
			\end{itemize}		 		 
		}
		\UCitem{Status}{
			\begin{itemize}
				\item Pendiente.
			\end{itemize}		 		  		 
		}
	\end{UseCase}
%	%-------------------------------------- COMIENZA descripción Trayectoria Principal
	\begin{UCtrayectoria}{Principal}
		\UCpaso[\UCactor] De la interfaz \IUref{UI2.3.1}{Enviar reporte}, selecciona el botón Generar y Enviar Reporte. \Trayref{A}\Trayref{B}\Trayref{C}.	
		\UCpaso[\UCsist] Muestra el mensaje en pantalla \MSGref{MSJ2.4}{Envío de reporte exitoso}.
	\end{UCtrayectoria}
	
%		%-------------------------------------- COMIENZA descripción Trayectoria Alternativa.
	
	\begin{UCtrayectoriaA}{A}{Cancelar reporte}
		\UCpaso[\UCsist] Enviara el mensaje en pantalla \MSGref{MSJ2.6}{Se ha cancelado la generación del reporte.}
		\UCpaso[\UCsist] Regresa al paso 1 de la trayectoria principal.
	\end{UCtrayectoriaA}	
	
	\begin{UCtrayectoriaA}{B}{Error de sistema}
		\UCpaso[\UCsist] Enviara el mensaje en pantalla \MSGref{MSJ2.8}{Se ha presentado un error de sistema al generar el reporte.}
		\UCpaso[\UCsist] Regresa al paso 1 de la trayectoria principal.
	\end{UCtrayectoriaA}
	
	\begin{UCtrayectoriaA}{C}{Error de envío}
		\UCpaso[\UCsist] Enviara el mensaje en pantalla \MSGref{MSJ2.7}{Se ha presentado un error de envío de reporte. Reintente más tarde.}
		\UCpaso[\UCsist] Regresa al paso 1 de la trayectoria principal.
	\end{UCtrayectoriaA}			
		
%-------------------------------------- TERMINA descripción del caso de uso.