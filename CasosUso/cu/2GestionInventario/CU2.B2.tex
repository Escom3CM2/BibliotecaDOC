% Copie este bloque por cada caso de uso:
%-------------------------------------- COMIENZA descripción del caso de uso.

	\begin{UseCase}{CU2.B2}{Generar Reporte de elementos escaneados}{
	   Permite mostrar la información escaneada a partir del CU2.B1, la cual será usada para generar un reporte con los detalles de cada uno de los elementos contenidos en las listas escaneadas del sistema.
	}
		\UCitem{Versión}{1}
		\UCitem{Actor}{Empleado de Procesos Técnicos}
		\UCitem{Propósito}{Generar un reporte con los elementos escaneados.}
		\UCitem{Entradas}{
			Lista de elementos con los siguientes datos, según sea el caso:
			\begin{itemize}
				\item ->LIBRO
				\item ISBN	
				\item Titulo	
				\item Autor(es)	
				\item Editorial	
				\item Fecha de Publicación	
				\item Numero de Páginas	
				\item Estado
				\item Precio	

				\item ->TT	
				\item Número de TT	
				\item Titulo	
				\item Autor(es)	
				\item Fecha de Presentación	
				\item Director(es)
	
				\item ->MATERIAL AUDIOVISUAL
				\item ID	
				\item Titulo	
				\item Autor(es)	
				\item Duracion	
				\item Fecha de Publicacion	
				\item Estado
				\item Precio
	
				\item ->EQUIPO DE CÓMPUTO	
				\item ID
				\item Marca
				\item Modelo
				\item Estado
				\item Precio
		
			\end{itemize}				
		}
		\UCitem{Salidas}{
		\begin{itemize}
			\item Mensajes para el usuario:
				\MSGref{MSJ2.4}{Envío de reporte exitoso}
				\MSGref{MSJ2.5}{Registros insuficientes.}
				\MSGref{MSJ2.6}{Reporte cancelado.}
				\MSGref{MSJ2.7}{Error de envío.}
			\item Reporte de elementos escaneados.
		\end{itemize}				
		}
		\UCitem{Precondiciones}{ 
			\begin{itemize}
				\item Ingresar al sistema con una credencial de procesos técnicos.
				\item Debe existir al menos un elemento escaneado en cada categoría para generar un reporte.
			\end{itemize}
		}
		\UCitem{Postcondiciones}{
		\begin{itemize}
		\item El actor/usuario podrá consultar el reporte de manera digital.
		\end{itemize}				
		}
		\UCitem{Autor(es)}{
		\begin{itemize}
		 \item Tejeda Martínez José Miguel
		\end{itemize}		 		 
		}

	\end{UseCase}
		%-------------------------------------- COMIENZA descripción Trayectoria Principal
	\begin{UCtrayectoria}{Principal}
	\UCpaso[\UCactor] Selecciona el tipo de material (Libro, Audiovisual o Equipo de cómputo) en el menú lateral de la pantalla \IUref{UI2.B1.1}{Inicio}.
	 \UCpaso[\UCactor]Desde la interfaz \IUref{UI2.B2.1}{Escanear libro}, en el caso de haber elegido Libro, la interfaz \IUref{UI2.B2.2}{Escanear equipo audiovisual} en caso de haber elegido audiovisual o \IUref{UI2.B2.3}{Escanear equipo de cómputo}, en caso de haber elegido equipo de cómputo, el usuario selecciona el botón "Visualizar Reporte".
	   \UCpaso[\UCsist] Verifica la  \BRref{RN03}{Registros mínimos para reporte}.[Trayectoria A]
	   \UCpaso[\UCsist]Muestra la interfaz \IUref{UI2.B2.4}{Reporte de Libros escaneados}, en el caso de haber elegido Libro, la interfaz \IUref{UI2.B2.5}{Reporte de material audiovisual} en caso de haber elegido audiovisual o \IUref{UI2.B2.6}{Reporte de equipo de cómputo}, en caso de haber elegido equipo de cómputo.
	    \UCpaso[\UCactor] Selecciona el botón "Generar y Enviar Reporte". [Trayectoria B][Trayectoria C]
	    \UCpaso[\UCsist] Muestra el mensaje en pantalla \MSGref{MSJ2.4}{Envío de reporte exitoso}.
	\end{UCtrayectoria}

			%-------------------------------------- COMIENZA descripción Trayectoria Alternativa.
		\begin{UCtrayectoriaA}{A}{Registros insuficientes para el reporte}
			\UCpaso[\UCsist] Enviara el  mensaje en pantalla \MSGref{MSJ2.5}{No existen suficientes registros para preparar un reporte.}
			\UCpaso[\UCsist] Regresa al paso 1 de la trayectoria principal.
		\end{UCtrayectoriaA}
		
		
		\begin{UCtrayectoriaA}{B}{Cancelar reporte}
			\UCpaso[\UCsist] Enviara el mensaje en pantalla \MSGref{MSJ2.6}{Se ha cancelado la generación del reporte.}
			\UCpaso[\UCsist] Regresa al paso 1 de la trayectoria principal.
		\end{UCtrayectoriaA}	
		
		\begin{UCtrayectoriaA}{C}{Error de envío}
			\UCpaso[\UCsist] Enviara el mensaje en pantalla \MSGref{MSJ2.7}{Se ha presentado un error de envío de reporte. Reintente más tarde.}
			\UCpaso[\UCsist] Regresa al paso 1 de la trayectoria principal.
		\end{UCtrayectoriaA}			
		
		
		
		
%-------------------------------------- TERMINA descripción del caso de uso.