% Copie este bloque por cada caso de uso:
%-------------------------------------- COMIENZA descripción del caso de uso.

	\begin{UseCase}{CU2.B1}{Escanear Códigos de Barras}{
	   Permite ingresar al sistema el código de barras de un libro, material audiovisual o equipo de cómputo, con el propósito de almacenar una lista de estos elementos escaneados, sirviendo posteriormente para actividades informativas sobre los detalles de los elementos existentes.
	}
		\UCitem{Versión}{1}
		\UCitem{Actor}{Empleado de Procesos Técnicos}
		\UCitem{Propósito}{Generar una lista con los elementos escaneados.}
		\UCitem{Entradas}{
		\begin{itemize}
		   \item Código de barras del elemento seleccionado.
		\end{itemize}				
		}
		\UCitem{Salidas}{
			Lista de elementos con los siguientes datos, según sea el caso:
			\begin{itemize}
				\item ->LIBRO
				\item ISBN	
				\item Titulo	
				\item Autor(es)	
				\item Editorial	
				\item Fecha de Publicación	
				\item Numero de Páginas	
				\item Estado
				\item Precio	

				\item ->TT	
				\item Número de TT	
				\item Titulo	
				\item Autor(es)	
				\item Fecha de Presentación	
				\item Director(es)
	
				\item ->MATERIAL AUDIOVISUAL
				\item ID	
				\item Titulo	
				\item Autor(es)	
				\item Duracion	
				\item Fecha de Publicacion	
				\item Estado
				\item Precio
	
				\item ->EQUIPO DE CÓMPUTO	
				\item ID
				\item Marca
				\item Modelo
				\item Estado
				\item Precio
		
			\end{itemize}				
		}
		\UCitem{Precondiciones}{ 
			\begin{itemize}
				\item Ingresar al sistema con una credencial de procesos técnicos.
				\item El material a escanear debe haber sido registrado previamente.
			\end{itemize}
		}
		\UCitem{Postcondiciones}{
		\begin{itemize}
		\item Los elementos permanecerán en una lista de elementos escaneados
		\end{itemize}				
		}
		\UCitem{Autor(es)}{
		\begin{itemize}
		 \item Tejeda Martínez José Miguel
		\end{itemize}		 		 
		}

	\end{UseCase}
		%-------------------------------------- COMIENZA descripción Trayectoria Principal
	\begin{UCtrayectoria}{Principal}
	  \UCpaso[\UCactor] Selecciona el tipo de material (Libro, Audiovisual o Equipo de cómputo) en el menú lateral de la pantalla \IUref{UI2.B1.1}{Inicio}.
	 \UCpaso[\UCsist]Muestra la interfaz \IUref{UI2.B1.2}{Escanear libro}, en el caso de elegir Libro, la interfaz \IUref{UI2.B1.3}{Escanear equipo audiovisual} en caso de elegir audiovisual o \IUref{UI2.B1.4}{Escanear equipo de cómputo}, en caso de elegir equipo de cómputo.
	  \UCpaso[\UCactor] Ingresa el código de barras del elemento a escanear.
	  \UCpaso[\UCactor] Da click en el botón "Agregar".
	   \UCpaso[\UCsist] Verifica la  \BRref{RN01}{Campos incompletos}.[Trayectoria A]
	    \UCpaso[\UCsist] Verifica la \BRref{RN02}{Material no registrado}.[Trayectoria B]
	    \UCpaso[\UCsist] Muestra el mensaje \MSGref{MSJ2.1}{Registro exitoso}
	    \UCpaso[\UCsist] Muestra la información del material correspondiente en la tabla inferior al botón "Agregar".
	\end{UCtrayectoria}

			%-------------------------------------- COMIENZA descripción Trayectoria Alternativa.
		\begin{UCtrayectoriaA}{A}{Campos incompletos}
			\UCpaso[\UCsist] Enviara el  mensaje en pantalla \MSGref{MSJ2.2}{Datos incompletos. Favor de completar todos los campos de entrada obligatorios.}
			\UCpaso[\UCsist] Regresa al paso 3 de la trayectoria principal.
		\end{UCtrayectoriaA}
		
		
		\begin{UCtrayectoriaA}{B}{Material no registrado}
			\UCpaso[\UCsist] Enviara el mensaje en pantalla \MSGref{MSJ2.3}{El material a escanear no está registrado.}
			\UCpaso[\UCsist] Regresa al paso 3 de la trayectoria principal, vaciando el campo de texto de entrada.
		\end{UCtrayectoriaA}	
		
		
		
		
		
		
%-------------------------------------- TERMINA descripción del caso de uso.