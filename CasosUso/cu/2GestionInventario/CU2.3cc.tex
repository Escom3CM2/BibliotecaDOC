% Copie este bloque por cada caso de uso:
%-------------------------------------- COMIENZA descripción del caso de uso.

\begin{UseCase}{CU2.3cc}{Dar de baja equipo audiovisual}{
	Este caso de uso se podrá eliminar un equipo audiovisual que se requiera dar de baja seleccionándolo desde el catálgo, y asi actualizar el numero de existencias
}
		\UCitem{Versión}{0.2}
		\UCitem{Actor}{Procesos Técnicos}
		\UCitem{Propósito}{La información del equipo audiovisual será puesta en un estado de borrado del catálogo para tener un control de las salidas de libros del inventario}
		\UCitem{Entradas}{
			\begin{itemize}
		   		\item Selección del equipo audiovisual deseado
		  	\end{itemize}					
		}
		\UCitem{Salidas}{
			\begin{itemize}
				\item Mensajes para el usuario
				\begin{itemize}
		   			\item \MSGref{MSJ2.3ca1}{Eliminación existosa}
		   			\item \MSGref{MSJ2.3ca2}{Eliminación de material}
		   			\item \MSGref{MSJ2.1b3}{Error en conexión}
		   		\end{itemize}
		  	\end{itemize} 
		}
		\UCitem{Precondiciones}{ 
			\begin{itemize} 
				\item Se debió haber efectuado el \UCref{CU2.3ac}
				\item El sistema debe de estar conectado a la base de datos
			\end{itemize}
		}
		\UCitem{Postcondiciones}{
			\begin{itemize}
				\item Se actualiza la base de datos sin los datos del equipo audiovisual eliminado
			\end{itemize}			
		}
		\UCitem{Autor}{
			\begin{itemize}
				\item Martínez Vilchis Juan Moisés
			\end{itemize}					
		}
		\UCitem{Revisó}{
			\begin{itemize}
				\item Rodrigo Burciaga
			\end{itemize}					
		}
		\UCitem{Estatus}{
			\begin{itemize}
				\item En revisión [24 / Mayo / 2017]
			\end{itemize}					
		}
\end{UseCase}

%----------------- COMIENZA la descripción de Trayectoria Principal

\begin{UCtrayectoria}{Principal}
		\UCpaso[\UCactor] Selecciona uno de los equipos audiovisuales mostrados en la tabla 'Seleccionar equipo audiovisual' de la pantalla \IUref{IU2.3cc}{ConsultarCatEqAudio}
		\UCpaso[\UCactor] Presiona el botón \IUbutton{Eliminar Eq Audiovisual Seleccionado} \Trayref{A}
		\UCpaso[\UCsist] Despliega la pantalla \IUref{IU2.3cc1}{EliminarEqAudio}
		\UCpaso[\UCsist] Mostrará los datos del libro seleccionado en la pantalla [ID, Titulo, Autores, Estado, Disponibilidad]
		\UCpaso[\UCactor] Confirma la operación presionando el botón \IUbutton{Confirmar} \Trayref{Cancelar}
		\UCpaso[\UCsist] Verifica que se cumpla la regla de negocio \BRref{RN2.3ca1}{Eliminación de material} \Trayref{B}
		\UCpaso[\UCactor] Actualiza el EstadoBorrado del libro con el valor de 1 \Trayref{A}
		\UCpaso[\UCsist] Muestra el mensaje \MSGref{MSJ2.3ca1}{Eliminación existosa}
		\UCpaso[\UCactor] Confirma operación haciendo click en el botón \IUbutton{Confirmar} en la ventana emergente que muestra
		\UCpaso[\UCsist] Despliega la pantalla \IUref{IU2.3cc}{ConsultarCatEqAudio}
\end{UCtrayectoria}


%--------------- COMIENZAN las Trayectorias Alternativas

%---------- TRAYECTORIA A

\begin{UCtrayectoriaA}{A}{Falló la conexion en la base de datos}
			\UCpaso[\UCsist] El sistema mostrará el mensaje \MSGref{MSJ2.1b3}{Error en conexión}
			\UCpaso[\UCsist] Regresa al paso 1 de la trayectoria principal \Trayref{Principal}
\end{UCtrayectoriaA}

%---------- TRAYECTORIA B

\begin{UCtrayectoriaA}{B}{Se viola la regla de negocio \BRref{RN2.3ca1}{Eliminación de material}}	
			\UCpaso[\UCsist] Informa al actor mostrando el mensaje de error \MSGref{MSJ2.3ca2}{Eliminación de material}
			\UCpaso[\UCsist] Regresa al paso 4 de la trayectoria principal \Trayref{Principal}
\end{UCtrayectoriaA}

%---------- TRAYECTORIA Cancelar

\begin{UCtrayectoriaA}{Cancelar}{El actor presiona el botón \IUbutton{Cancelar}}
			\UCpaso[\UCsist] Regresa al paso 1 de la trayectoria principal \Trayref{Principal} 
\end{UCtrayectoriaA}