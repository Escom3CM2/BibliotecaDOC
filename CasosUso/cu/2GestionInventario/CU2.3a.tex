% Copie este bloque por cada caso de uso:
%-------------------------------------- COMIENZA descripción del caso de uso.

	\begin{UseCase}{CU2.3.a}{Consultar material}{
	   El usuario deberá ingresar los datos del material a consultar que previamente ha sido registrado en el sistema, posteriormente el sistema mostrará una tabla con la información del material en cuestión.
	}
		\UCitem{Versión}{1}
		\UCitem{Actor}{Usuario}
		\UCitem{Propósito}{Mostrar al usuario la información del material que desee consultar.}
		\UCitem{Entradas}{
		\begin{itemize}
		   \item ISBN del Libro
		   \item Número de TT 
		   \item ID Audiovisual
		   \item Titulo del material audiovisual
		   \item Número de computadora
		\end{itemize}				
		}
		\UCitem{Salidas}{
		\begin{itemize}
			\item ISBN	
			\item Titulo	
			\item Autor(es)	
			\item Editorial	
			\item Fecha de Publicación	
			\item Numero de Páginas	
			\item Existencias	
			\item Edición	
			\item Precio	
			\item Existencias

			\item Número de TT	
			\item Titulo	
			\item Autor(es)	
			\item Fecha de Presentación	
			\item Director(es)

			\item ID	
			\item Titulo	
			\item Autor(es)	
			\item Fecha de Publicacion	
			\item Duracion	
			\item Existencias	
			\item Precio	
			\item Tipo

			\item Numero de serie monitor 
			\item Precio monitor
			\item Numero de serie CPU
			\item Numero de serie teclado
			\item Precio teclado
			\item Numero de serie Mouse
			\item Precio Mouse						
	
		\end{itemize}				
		}
		\UCitem{Precondiciones}{ 
			\begin{itemize}
				\item El material a consultar debe haber sido registrado previamente
			\end{itemize}
		}
		\UCitem{Postcondiciones}{
		\begin{itemize}
		\item Pendiente
		\end{itemize}				
		}
		\UCitem{Autor(es)}{
		\begin{itemize}
		 \item Cortes Frias Diego Antonio
		\end{itemize}		 		 
		}

	\end{UseCase}
		%-------------------------------------- COMIENZA descripción Trayectoria Principal
	\begin{UCtrayectoria}{Principal}
	\UCpaso[\UCactor] Ingresa al sistema mediante el enlace a la aplicacion web
	 \UCpaso[\UCactor] Selecciona la intefaz \IUref{UI2.8}{Modificar o eliminar}
	  \UCpaso[\UCactor] Selecciona el tipo de material (Libro, TT, Audiovisual o Equipo de cómputo).
	 \UCpaso[\UCsist]Muestra la interfaz \IUref{UI2.9}{Consultar catálogo del libro}, en el caso de elegir Libro, la interfaz \IUref{UI2.10}{Consultar catálogo de TT} en caso de elegir TT, la interfaz \IUref{UI2.11}{Consultar catálogo de equipo audiovisual} en caso de elegir audiovisual o \IUref{UI2.12}{Consultar catálogo de equipo de cómputo}, en caso de elegir equipo de cómputo
	  \UCpaso[\UCactor] ingresa los datos correspondientes
	  \UCpaso[\UCactor] Da click en consultar
	   \UCpaso[\UCsist] Verifica la  \BRref{RN42}{Campos sin llenar}.[Trayectoria A]
	    \UCpaso[\UCsist] Verifica la \BRref{RN39}{Material no registrado}.[Trayectoria B]
	    \UCpaso[\UCsist] Muestra la información del material correspondiente.


	\end{UCtrayectoria}
			%-------------------------------------- COMIENZA descripción Trayectoria Alternativa.
		\begin{UCtrayectoriaA}{A}{Campos sin llenar}
			\UCpaso[\UCsist] Enviara el  mensaje en pantalla \MSGref{MSJ1.2}{Datos incompletos. Favor de completar todos los campos de entrada marcados con un *}
			\UCpaso[\UCsist] Regresa al paso 5 de la trayectoria principal conservando los datos de los campos que si fueron llenados con algún valor.
		\end{UCtrayectoriaA}
		
		
		\begin{UCtrayectoriaA}{B}{Material no registrado}
			\UCpaso[\UCsist] Enviara el mensaje en pantalla \MSGref{MSJ1.3}{El material a consultar no está registrado}
			\UCpaso[\UCsist] Regresa al paso 5 de la trayectoria principal	
		\end{UCtrayectoriaA}	
		
		
		
		\begin{UCtrayectoriaA}{G}{Los datos no coinciden}
			\UCpaso[\UCsist] Enviará el mensaje  \MSGref{MSG06}{Los datos ingresados en CURP no son correctos}
			\UCpaso[\UCsist] Se regresará al paso 4 de la trayectoria principal conservando todos los datos a excepción de la CURP.

		\end{UCtrayectoriaA}
		
		
		
		
		
		
%-------------------------------------- TERMINA descripción del caso de uso.