% Copie este bloque por cada caso de uso:
%-------------------------------------- COMIENZA descripción del caso de uso.

	\begin{UseCase}{CU3.5}{Devolver Material}{
		El Bibliotecario podrá registrar las devoluciones de los materiales que el lector tanto interno como externo hagan, asi como las observaciones fisicas y en dado caso las multas que estos generen
		}
		\UCitem{Actor}{Bibliotecario}
		\UCitem{Propósito}{Registrar las devoluciones que hagan los lectores}
		\UCitem{Entradas}{Ninguna}
		\UCitem{Salidas}{Actualización de la BD}
		\UCitem{Precondiciones}{Ninguna}
		\UCitem{Postcondiciones}{Ninguna}
		\UCitem{Autor}{Omar Patrón y Cortés Pérez Edy}
		\UCitem{Estatus}{Revision}
	\end{UseCase}
		%-------------------------------------- COMIENZA descripción Trayectoria Principal
		\begin{UCtrayectoria}{Principal}
		\UCpaso[\UCactor] Pulsa \IUbutton{Devoluciones} del menú del \UCref{CU3.2} o \UCref{CU3.3}
		\UCpaso[\UCsist] Busca en la BD los prestamos en estado de Pendiente asociados al Lector [Trayectoria A].
		\UCpaso[\UCsist] Muestra la pantalla \IUref{IU3.5}{Devoluciones}
		\UCpaso[\UCactor] Selecciona los materiales a devolver, en caso de que el material tenga maltrato fisico pulsa \IUbutton{Editar Observaciones} \Trayref{B}
		\UCpaso[\UCactor] Pulsa \IUbutton{Devolver}
		\UCpaso[\UCsist] Registra las devoluciones en la BD y cambia el estado de los materiales a Disponible
		\UCpaso[\UCsist] Elimina los materiales de la tabla de devoluciones, si la tabla ya no tiene elementos nos regresa al menu de acción del \UCref{CU3.2} o \UCref{CU3.3} segun sea el caso
			\end{UCtrayectoria}
			%-------------------------------------- COMIENZA descripción Trayectoria Alternativa.
			
		\begin{UCtrayectoriaA}{A}{Error al conectar a la base de datos}
			\UCpaso[\UCsist] Muestra el mensaje \MSGref{MSJ3.1}{Error al conectar a la BD}
			\UCpaso[\UCactor] Pulsa \IUbutton{OK}
			\UCpaso[\UCsist] Regresa al paso 2 de la trayectoria principal.
		\end{UCtrayectoriaA}
%-------------------------------------- 
		\begin{UCtrayectoriaA}{B}{Pulso el boton \IUbutton{Editar Observaciones} de la tabla Devoluciones}
			\UCpaso[\UCsist] Muestra en pantalla el cuadro de Observaciones.
			\UCpaso[\UCactor] Anota las observaciones de acuerdo a las caracteristicas fisicas que tenga el material en cuestión
			\UCpaso[\UCsist] Pulsa el \IUbutton{Aceptar}
			\UCpaso[\UCsist] Cambia las Observaciones de los materiales que fueron editadas por el lector
			\UCpaso[\UCactor] Pulsa \IUbutton{Generar Multa}
			\UCpaso[\UCsist] Registra las devoluciones en la BD y cambia el estado de los materiales a Disponible
			\UCpaso[\UCsist]	Redirige al \UCref{CU3.7}.
		\end{UCtrayectoriaA}
		TERMINA descripción del caso de uso.


