% Copie este bloque por cada caso de uso:
%-------------------------------------- COMIENZA descripción del caso de uso.

	\begin{UseCase}{CU3.9}{Consultar Préstamos de Material}{
		El jefe de Biblioteca podrá realizar una búsqueda de prestamos de Material, para obtener información del préstamo y lo hará por medio de filtros de busqueda.
	}
		\UCitem{Versión}{0.1}
		\UCitem{Actor}{Jefe de biblioteca}
		\UCitem{Propósito}{Obtener la información de los prestamos de material que fueron realizados dentro de la biblioteca de ESCOM.}
		\UCitem{Entradas}{Seleccionar una de las siguientes opciones:
			\begin{itemize}
				\item ID material
				\item Titulo
				\item Usuario
				\item Tipo de Material
				\item Fecha
			\end{itemize}
		}
		\UCitem{Salidas}{Una tabla con los datos del préstamo.}
		\UCitem{Precondiciones}{Ninguna}
		\UCitem{Postcondiciones}{Ninguna}
		\UCitem{Autor}{Cortés Pérez Edy}
		\UCitem{Estatus}{Revisión}
	\end{UseCase}
		%-------------------------------------- COMIENZA descripción Trayectoria Principal
	\begin{UCtrayectoria}{Principal}
		\UCpaso[\UCactor] Pulsa el botón \IUbutton{Consulta Prestamos} del menú de acción del perfil del Jefe de biblioteca.
		\UCpaso[\UCsist]Muestra la pantalla \IUref{IU3.9}{ConsultaPres}.
		\UCpaso[\UCactor]Selecciona uno de los filtros de búsqueda.
		\UCpaso[\UCactor]Ingresa los datos adecuados de la busqueda, correspondiente al tipo de filtro que selecciono.
		\UCpaso[\UCactor]Pulsa el botón \IUbutton{Buscar}
		\UCpaso[\UCsist]Verifica la conexion con la Base de Datos \Trayref{B}
		\UCpaso[\UCsist]Verifica la \BRref{R1.1}{Campos no nulos} \Trayref{A}
		\UCpaso[\UCsist]Verifica la \BRref{RN3.28}{Datos de la Consulta} \Trayref{C}
		\UCpaso[\UCsist]Busca en la BD en la tabla de prestamos todos los prestamos que concuerden con los datos que a metido el usuario, dependiendo de el filtro que se selecciono.
		\UCpaso[\UCsist] Muestra la pantalla \IUref{IU3.9a}{ResultadoPres}, la cual contiene la información de los resultados de la busqueda.
		\UCpaso[\UCsist] Muestra el mensaje \MSGref{MSJ3.9}{Busqueda Satisfactoria}
	\end{UCtrayectoria}
			%-------------------------------------- COMIENZA descripción Trayectoria Alternativa.
		\begin{UCtrayectoriaA}{A}{Hay Campos Vacios}
			\UCpaso[\UCsist] Muestra el mensaje \MSGref{MSJ3.7}{Campos Vacios}
			\UCpaso[\UCactor] Presiona el botón \IUbutton{OK}
			\UCpaso[\UCsist] Regresa al paso 3 de la trayectoria principal.
		\end{UCtrayectoriaA}


%--------------------------------------
		\begin{UCtrayectoriaA}{B}{Error al conectar a la base de datos}
			\UCpaso[\UCsist] Muestra el mensaje \MSGref{MSJ3.1}{Error al conectar en la BD}
			\UCpaso[\UCactor] Presiona el botón \IUbutton{OK}
			\UCpaso[\UCsist] Regresa al paso 3 de la trayectoria principal.
		\end{UCtrayectoriaA}
		
		\begin{UCtrayectoriaA}{C}{Los datos de la consulta son erroneos o no existen en la BD}
			\UCpaso[\UCsist] Muestra el mensaje \MSGref{MSJ3.10}{Datos Erróneos}
			\UCpaso[\UCactor] Presiona el botón \IUbutton{OK}
			\UCpaso[\UCsist] Regresa al paso 3 de la trayectoria principal.
		\end{UCtrayectoriaA}
		
TERMINA descripción del caso de uso.


