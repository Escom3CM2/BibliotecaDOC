% Copie este bloque por cada caso de uso:
%-------------------------------------- COMIENZA descripción del caso de uso.

	\begin{UseCase}{CU3.10}{Consultar Préstamo Interbibliotecario}{
		El jefe de Biblioteca podrá realizar una búsqueda de prestamos interbibliotecarios para saber cuantos formatos salieron de la biblioteca, que libros pidieron y a que bibliotecas se los pidieron, para obtener información del préstamo y lo hará con un filtro: por Lector, folio o titulo del libro
	}
		\UCitem{Versión}{0.1}
		\UCitem{Actor}{Jefe de biblioteca}
		\UCitem{Propósito}{Obtener la información de los prestamos interbibliotecarios que fueron realizados dentro de la biblioteca de ESCOM.}
		\UCitem{Entradas}{Seleccionar una de las siguientes opciones:
			\begin{itemize}
				\item Usuario
				\item Folio
				\item Titulo del libro
				\item Biblioteca
			\end{itemize}
		}
		\UCitem{Salidas}{Una tabla con los datos del préstamo.}
		\UCitem{Precondiciones}{Ninguna}
		\UCitem{Postcondiciones}{Ninguna}
		\UCitem{Autor}{Rodriguez Cervantes Arturo.}
		\UCitem{Estatus}{Revisión}
	\end{UseCase}
		%-------------------------------------- COMIENZA descripción Trayectoria Principal
	\begin{UCtrayectoria}{Principal}
		\UCpaso[\UCactor]Pulsa el botón \IUbutton{Consulta P. Interbibliotecario} de la barra de acción del perfil del jefe de biblioteca
		\UCpaso[\UCsist]Muestra la pantalla \IUref{IU3.10}{ConsultaPresInt}.
		\UCpaso[\UCactor]Selecciona uno de los métodos de búsqueda.
		\UCpaso[\UCactor]Ingresa el texto en la búsqueda, si selecciono Biblioteca solo selecciona la Biblioteca que desea buscar.
		\UCpaso[\UCactor]Pulsa el botón \IUbutton{Buscar}
		\UCpaso[\UCsist]Se conecta a la base de datos[trayectoria A]
		\UCpaso[\UCsist]Verifica la \BRref{RN1.1}{Campos no nulos} [trayectoria B]
		\UCpaso[\UCsist]Se buscan los prestamos conforme el filtro que se hizo previamente.
		\UCpaso[\UCsist]Muestra la pantalla \IUref{IU3.10a}{ResultadoPresInt} con los datos de los préstamos previamente filtrados
	\end{UCtrayectoria}
			%-------------------------------------- COMIENZA descripción Trayectoria Alternativa.
		\begin{UCtrayectoriaA}{A}{Erro al conectar a la base de datos}
			\UCpaso[\UCsist] Muestra el mensaje \MSGref{MSJ3.1}{Error al conectar en la BD.}
			\UCpaso[\UCactor] Presiona el botón \IUbutton{OK}
			\UCpaso[\UCsist] Regresa al paso 2 de la trayectoria principal.
		\end{UCtrayectoriaA}	
		
			%--------------------------------------
			
		\begin{UCtrayectoriaA}{B}{Campo vació}
			\UCpaso[\UCsist] Muestra el mensaje \MSGref{MSJ1.11}{ERROR. Completa todos los campos de entrada.}
			\UCpaso[\UCactor] Presiona el botón \IUbutton{OK}
			\UCpaso[\UCsist] Regresa al paso 2 de la trayectoria principal.
		\end{UCtrayectoriaA}

		
		
TERMINA descripción del caso de uso.


