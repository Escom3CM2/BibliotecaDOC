%-------------------------------------- COMIENZA descripción del caso de uso.

	\begin{UseCase}{CU3.2}{Verificar Lector Interno}{
		El Bibliotecario corroborará con la ayuda del sistema si el lector es apto para poder realizar cualquier proceso dentro de la biblioteca, con base a las validaciones que el sistema hara.
	}
		\UCitem{Versión}{0.4}
		\UCitem{Actor}{Bibliotecario}
		\UCitem{Propósito}{Comparar el estatus del lector con las validaciones para saber si se es apto para realizar los procesos que se le son permitidos.}
		\UCitem{Entradas}{ID Usuario}
		\UCitem{Salidas}{Ninguna.}
		\UCitem{Precondiciones}{Ninguna}
		\UCitem{Postcondiciones}{Ninguna}
		\UCitem{Autor}{Vega Camacho Enrique A.}
		\UCitem{Estatus}{Revisión}
		\UCitem{Revisión}{Cortés Pérez Edy}
	\end{UseCase}
		%-------------------------------------- COMIENZA descripción Trayectoria Principal
	\begin{UCtrayectoria}{Principal}
		\UCpaso[\UCactor] Clic en botón \IUbutton{Interno} del Menú Lateral del Perfil de Préstamos, \IUref{IU3.1}{GestionPrestamos}.
		\UCpaso[\UCsist] Muestra la \IUref{IU3.2}{InicioInterno}  
		\UCpaso[\UCactor] Ingresa el ID del Lector que desee realizar alguna operación dentro de la biblioteca.
		\UCpaso[\UCactor] Pulsar el botón \IUbutton{Verificar}. 
		\UCpaso[\UCsist] Verifica la conexión a la base de datos \Trayref{A} 
		\UCpaso[\UCsist] Verifica la \BRref{RN3.1}{Lector acreedor a Prestamo a Domicilio}  \Trayref{B}
		\UCpaso[\UCsist] Busca en la BD los prestamos asociados al lector
		\UCpaso[\UCsist] Verifica la \BRref{RN3.3}{Devoluciones atrasadas} \Trayref{D}
		\UCpaso[\UCsist] Verifica la \BRref{RN3.2}{Lector con Multas} \Trayref{E}
		\UCpaso[\UCsist] Verifica la \BRref{RN3.6}{Numero de Prestamos} \Trayref{F}
		\UCpaso[\UCsist] Verifica la \BRref{RN3.4}{Credencial Vigente} \Trayref{C}
		\UCpaso[\UCsist] Muestra la \IUref{IU3.2.1}{LectorInternoApto}
		\UCpaso[\UCsist] Muestra el \MSGref{MSJ3.16}{Lector Apto}
		\UCpaso[\UCactor] Pulsa el botón \IUbutton{OK} 
	\end{UCtrayectoria}

			%-------------------------------------- COMIENZA descripción Trayectoria Alternativa.

		\begin{UCtrayectoriaA}{A}{No se realizó una conexión a la base de datos con éxito.}
			\UCpaso[\UCsist] Muestra el mensaje \MSGref{MSJ3.1}{Error al conectar a la BD}
			\UCpaso[\UCactor] Presiona el botón \IUbutton{OK}
			\UCpaso[\UCsist] Regresa al punto 3 de la \Trayref{Principal}
		\end{UCtrayectoriaA}
%-------------------------------------		
		\begin{UCtrayectoriaA}{B}{No se encontró un Lector relacionado al ID ingresado.}
			\UCpaso[\UCsist] Se muestra el \MSGref{MSJ3.2}{Lector Inexistente.}
			\UCpaso[\UCactor] Se presiona el botón \IUbutton{OK}			
			\UCpaso[\UCsist]	Se regresa al punto 2 de la \Trayref{Principal}
		\end{UCtrayectoriaA}
%-------------------------------------- 
		\begin{UCtrayectoriaA}{C}{El lector no tiene credencial vigente (no es vigente cuando la credencial tiene mas de 3 meses de tramitarla)}
			\UCpaso[\UCsist] Se muestra el \MSGref{MSJ3.21}{Credencial no vigente}
			\UCpaso[\UCactor] Se presiona el botón \IUbutton{OK}			
			\UCpaso[\UCsist]	Se regresa al punto 2 de la \Trayref{Principal}
		\end{UCtrayectoriaA}	
%-------------------------------------- 
		\begin{UCtrayectoriaA}{D}{El lector tiene devoluciones pendientes.}
			\UCpaso[\UCsist] Los prestamos que tengan una Fecha de Devolución expirada se hara una diferencia con la Fecha actual y se iran sumando para al final sacar un dia total de atrasos, en el caso de los TT's se hara por hora de atraso, desde la hora y fecha que estan registradas para su devolución
			\UCpaso[\UCsist] Los días y horas de atraso que de como resultado se multiplicaran por 6
			\UCpaso[\UCsist] Muestra la \IUref{IU3.2.2}{LectorDevolucionesPendientes} donde se mostrara una tabla con todos los materiales prestamos que tiene el lector, los materiales que tengan atraso seran sombreados en rojo y debajo de la tabla se mostrara un total parcial de la multa por Devolución Atrasada
			\UCpaso[\UCsist] Muestra el \MSGref{MSJ3.6}{Lector con Devoluciones Atrasadas}
			\UCpaso[\UCactor] Se presiona el botón \IUbutton{OK}
			\UCpaso[\UCactor]	Selecciona los materiales que se regresarán
			\UCpaso[\UCactor]	Pulsa el botón \IUbutton{Generar Multa}
			\UCpaso[\UCsist]	Registra la devolución de los Materiales en la BD, cambiando su estado a Disponible
			\UCpaso[\UCsist]	Redirige al \UCref{CU3.7}.
		\end{UCtrayectoriaA}
%--------------------------------------
		\begin{UCtrayectoriaA}{E}{El lector tiene multas pendientes por pagar}
			\UCpaso[\UCsist] Busca en la BD las multas asociadas al Lector
			\UCpaso[\UCsist] Muestra la \IUref{IU3.2.3}{LectorMultasPendientes}, donde se muestran en una tabla las multas Sin pagar que tiene el lector en cuestión
			\UCpaso[\UCsist] Se muestra el \MSGref{MSJ3.5}{Lector con Multas}
			\UCpaso[\UCactor] Se presiona el botón \IUbutton{OK}
			\UCpaso[\UCactor] Selecciona las multas que se desea hacer una operación con ellas
			\UCpaso[\UCactor] Pulsa alguno de los 3 botones \IUbutton{Pagar Multa} paso 7 Trayectoria E, \IUbutton{Cancelar Multa} \Trayref{G}, \IUbutton{Imprimir} \Trayref{H}
			\UCpaso[\UCsist] Cambia el estado de la multa a Pagada y se quita de la tabla de Multas Pendientes
			\UCpaso[\UCsist] Muestra el \MSGref{MSJ3.12}{Pago de Multa Exitoso}
			\UCpaso[\UCactor] Pulsa el \IUbutton{OK}
			\UCpaso[\UCsist] Si ya no ahi multas sin pagar asociadas al lector regresa al Paso 11 de la \Trayref{Principal}, si existen todavia Multa pendientes paso 5 de la \Trayref{E}
		\end{UCtrayectoriaA}
%--------------------------------------
		\begin{UCtrayectoriaA}{F}{El Lector Alcanzo el limite de prestamos permitidos}
			\UCpaso[\UCsist] Muestra el \MSGref{MSJ3.14}{Limite de prestamos alcanzado}
			\UCpaso[\UCactor] Pulsa el \IUbutton{OK}
			\UCpaso[\UCsist] Redirecciona al \UCref{CU3.5}
		\end{UCtrayectoriaA}
		%--------------------------------------
		\begin{UCtrayectoriaA}{G}{Pulso el \IUbutton{Cancelar Multa} en la tabla de Multas Pendientes}
			\UCpaso[\UCsist] Verifica la \BRref{RN3.26}{Cancelar Multa}\Trayref{I}
			\UCpaso[\UCsist] Al total de la multa se le resta el costo total del material que genero la multa por Perdida
			\UCpaso[\UCsist] Se verifica la fecha que se tenia que hacer la Devolución del Material en Cuestión y se hace la diferencia con la fecha actual, multiplicando el resultado por 6
			\UCpaso[\UCsist] Al total de la multa que previamente se le resto el costo del material se le suma el total que arrojo el paso anterior y se guarda, si el total de la multa es igual a 0 la multa se cancela por completo y se borrara de la tabla Paso 6 \Trayref{G}
			\UCpaso[\UCsist] Se modifica el total de la multa en la tabla y se quita el concepto de Perdida en su lugar se pone el concepto de Devolución Tardía
			\UCpaso[\UCsist] Muestra el \MSGref{MSJ3.18}{Cancelacion de Multa Exitosa}
			\UCpaso[\UCactor] Pulsa el \IUbutton{OK}, paso 5 \Trayref{E}
		\end{UCtrayectoriaA}
		%--------------------------------------
		\begin{UCtrayectoriaA}{H}{Pulso el \IUbutton{Imprimir} en la tabla de Multas Pendientes}
			\UCpaso[\UCsist] Se manda a la cola de impresión las multas seleccionadas con el formato \IUref{IU3.3.3a}{FormatoMultas}
			\UCpaso[\UCsist] Muestra el \MSGref{MSJ3.19}{Impresion de Multa}
			\UCpaso[\UCactor] Pulsa el \IUbutton{OK}, paso 5 \Trayref{E}
		\end{UCtrayectoriaA}
		
		\begin{UCtrayectoriaA}{I}{Se selecciona una multa con concepto de multa diferente de Perdida}
			\UCpaso[\UCsist] Muestra el \MSGref{MSJ3.20}{Error Cancelacion de Multa}
			\UCpaso[\UCactor] Pulsa el \IUbutton{OK}, paso 5 \Trayref{D}
		\end{UCtrayectoriaA}
	
TERMINA descripción del caso de uso.