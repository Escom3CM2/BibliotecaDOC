% Copie este bloque por cada caso de uso:
%-------------------------------------- COMIENZA descripción del caso de uso.

	\begin{UseCase}{CU3.4}{Registrar Prestamo de Material}{
		El Bibliotecario podrá registrar en el sistema los prestamos que se hagan en la biblioteca tanto a lectores internos como a lectores externos.
	}
		\UCitem{Versión}{0.1}
		\UCitem{Actor}{Bibliotecario}
		\UCitem{Propósito}{Registrar los Prestamos que se hagan en la biblioteca.}
		\UCitem{Entradas}{El ID del material, o el ID y el folio del lector Externo}
		\UCitem{Salidas}{Actualización de la BD}
		\UCitem{Precondiciones}{Ninguna}
		\UCitem{Postcondiciones}{Ninguna}
		\UCitem{Autor}{Rodriguez Cervantes Arturo.}
		\UCitem{Estatus}{Revisión}
	\end{UseCase}
		%-------------------------------------- COMIENZA descripción Trayectoria Principal
	\begin{UCtrayectoria}{Principal}
		\UCpaso[\UCactor] Pulsa el botón \IUbutton{Prestamo} del menú de acción del \UCref{CU3.2} o del \UCref{CU3.3}
		\UCpaso[\UCsist] Verifica si la solicitud es de un Lector Externo o Interno
		\UCpaso[\UCsist] Muestra la \IUref{IU3.4}{PrestamoInterno} si es lector interno, si es lector externo \Trayref{A}
		\UCpaso[\UCactor] Introduce el ID del material que se va a prestar.
		\UCpaso[\UCactor] Selecciona el tipo de material que es: CD, LIBRO o TT.
		\UCpaso[\UCactor] Pulsa el botón \IUbutton{Registrar Material}
		\UCpaso[\UCsist] Verifica la \BRref{RN3.6}{Numero de Prestamos} \Trayref{B}
		\UCpaso[\UCsist] Verifica la \BRref{RN3.10}{Numero de Ejemplares} \Trayref{F}
		\UCpaso[\UCsist] Busca el material en la BD y jala los datos requeridos
		\UCpaso[\UCsist] Verifica el tipo de material que se esta prestando, si el material es un TT \Trayref{C}.
		\UCpaso[\UCsist] Establecera la Fecha de Devolución como 5 habiles posteriores a la fecha actual.
		\UCpaso[\UCsist] Pondra los datos del prestamo en la tabla de Registro de Material.
		\UCpaso[\UCactor] Pulsa el botón \IUbutton{Registrar} paso 11 \Trayref{Principal} o selecciona algun Material de la Tabla de Registro de Material \Trayref{D}
		\UCpaso[\UCsist] Registra el Prestamo generando el ID del Prestamo y asociando los materiales que aparezcan en la tabla al lector y la fecha y hora de devolución
		\UCpaso[\UCsist] Muestra el \MSGref{MSJ3.3}{Préstamo Exitoso}
		\UCpaso[\UCactor] Presiona el botón \IUbutton{OK}
		\UCpaso[\UCsist] Paso 12 \UCref{CU3.2}.
	\end{UCtrayectoria}
			%-------------------------------------- COMIENZA descripción Trayectoria Alternativa.
		\begin{UCtrayectoriaA}{A}{El lector que hara la solicitud es un lector externo}
			\UCpaso[\UCsist] Verifica la \BRref{RN3.17}{Limite de Préstamos Interbibliotecarios a Lector} \Trayref{G}
			\UCpaso[\UCsist] Muestra \IUref{IU3.4a}{PrestamoExterno}
			\UCpaso[\UCactor] Introduce el Folio del formato y el ID del libro
			\UCpaso[\UCsist] Registra el prestamo generando un ID de prestamo, asociando el libro al lector externo en cuestión y la fecha de devolución
			\UCpaso[\UCsist] Muestra el \MSGref{MSJ3.3}{Préstamo Exitoso}
			\UCpaso[\UCactor] Presiona el botón \IUbutton{OK}
			\UCpaso[\UCsist] Paso 12 \UCref{CU3.3}.
		\end{UCtrayectoriaA}
%-------------------------------------- 
		\begin{UCtrayectoriaA}{B}{El lector tiene en calidad de prestamo el limite de un tipo de material}
			\UCpaso[\UCsist] Muestra el mensaje \MSGref{MSJ3.14}{Limite de préstamos alcanzado.}
			\UCpaso[\UCactor] Presiona el botón \IUbutton{OK}
			\UCpaso[\UCsist] Paso 3 de la \Trayref{Principal}.
		\end{UCtrayectoriaA}
%--------------------------------------
		\begin{UCtrayectoriaA}{C}{El material a registrar en un TT}
			\UCpaso[\UCsist] Establece la Fecha y hora de devolución como un hora posterior a la fecha y hora actual
			\UCpaso[\UCsist] Paso 11 de la \Trayref{Principal}.
		\end{UCtrayectoriaA}
%--------------------------------------
		\begin{UCtrayectoriaA}{D}{El lector selecciono un material de la tabla}
			\UCpaso[\UCsist] Pulsa el \IUbutton{Editar Observaciones} o el \IUbutton{Remover}
			\UCpaso[\UCsist] Abre el cuadro de edicion y donde se encuentran escritas las caracteristicas fisicas que tiene el material que se selecciono
			\UCpaso[\UCactor] Edita las observaciones respecto a las caracteristicas fisicas del material seleccionado
			\UCpaso[\UCactor] Pulsa el \IUbutton{Aceptar}
			\UCpaso[\UCsist] Guarda las nuevas observaciones relacionadas al Material
			\UCpaso[\UCsist] Paso 11 \Trayref{Principal}.
		\end{UCtrayectoriaA}
		
		\begin{UCtrayectoriaA}{E}{Pulso el \IUbutton{Remover} de la tabla de Registro de Material}
			\UCpaso[\UCsist] Quita el Material seleccionado de la tabla Registro de material
			\UCpaso[\UCsist] Paso 11 \Trayref{Principal}.
		\end{UCtrayectoriaA}
			
		\begin{UCtrayectoriaA}{F}{El lector ya registro otro mismo ejemplar en prestamo, o el lector ya tenia previamente el ejemplar en calidad de prestamo}
			\UCpaso[\UCsist] Muestra el \MSGref{MSJ3.22}{Numero de Ejemplares}
			\UCpaso[\UCactor] Pulsa \IUbutton{OK}
			\UCpaso[\UCsist] Paso 2 \UCref{CU3.5}.
		\end{UCtrayectoriaA}
			
		\begin{UCtrayectoriaA}{G}{El lector tiene el limite permitido de prestamos interbibliotecarios}
			\UCpaso[\UCsist] Muestra el \MSGref{MSJ3.14}{Limite de Préstamos Alcanzado}
			\UCpaso[\UCactor] Pulsa \IUbutton{OK}
			\UCpaso[\UCsist] Paso 2 \UCref{CU3.5}.
		\end{UCtrayectoriaA}
TERMINA descripción del caso de uso.


